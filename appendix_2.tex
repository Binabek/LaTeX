\chapter{Second Generation Danilov-Nagaitsev Insert Magnet} \label{apx:dnv2}

The second iteration of the DN integrable insert makes a few different design decisions from the first. The largest change is a reduction from 18 equally spaced lenses to 11 in an equi-phase spaced configuration. As discussed in \ref{sec:dnSims} we see similar quality of conservation of the analytically predicted invariants in this configuration with fewer elements. In this configuration, the central elements are more closely packed, and the relative integrated strength of the elements becomes the same. Figure \ref{fig:dnv2int} shows the integration of the nominal potential with the equi-phase elements. The dense central packing requires differently shaped individual lenses, with shorter magnets in the center.

\begin{figure}
	\centering
	\includegraphics[width=0.6\linewidth]{./appendix_figures/dnIntPiecewiseEquiPhi.pdf}
	\caption{Integration steps for equi-phase insert}
	\label{fig:dnv2int}
\end{figure}

The other major change in the insert design is a signifcantly icreased physical aperture. There are two ways to accomplish this, the first is increasing the geometric DN c-parameter. This naturally increases the radius of the good field region, but also effectively reduces the nonlinearities the beam is sensitive to. In the case of the new insert, the c-parameter is increased to $c=0.014 \sqrt{\mathrm{m}}$. The other adjustment is changing the contour in the magnetic potential used for the pole face. Contours further from the origin can be selected, which naturally increases the minimum aperture. For the second iteration of the magnet, two contours were used. For magnets near the center, a contour at a magnetic scalar potential of 0.79, normalized by the rigidity $B\rho$. This is to maximize the avialable aperture. For magnets further out on the edge, a scalar potential of 0.653 was used as the aperture restrictions are lesser here in the equi-phase configuration. Finally, the design of the vacuum chamber was changed from a complicated smootly tapering structure to stepped cylinders of constant radius. As of writing this dissertation, the vacuum chamber is still being manufactured, so the exact final dimensions cannot be verified.
