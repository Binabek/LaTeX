%% !TEX program = XeLaTeX

%\documentclass{report}
\documentclass{msu-thesis}

\usepackage{biblatex}
\usepackage{amsmath}
\usepackage{graphicx}
\usepackage{amsfonts}
%\usepackage{unicode-math}%, neccesary for vector graphic lowercase mathbb e, needs to use XeTex

%add siunitx package for scientific notation, set spacing to tight so powers of 10 fit well
\usepackage{siunitx}
\sisetup{tight-spacing=true} 

%adds bookmarks and external linking hopefully
%\usepackage[bookmarks]{hyperref}

%add subfig package for side-by-side figures
%\usepackage{subfig}

\addbibresource{z_chapter_2.bib}

\begin{document}

\frontmatter

\title{Experimental Studies of Nonlinear Integrable Optics with Elliptic Potentials}
%\author{John Wieland}
%\date{December 7, 1941}
%\fieldofstudy{Physics}
%
%\maketitlepage

\abstract{Stable transport of charged particle beams is the core challenge in designing and operating accelerators. The current paradigm in transverse focusing consists of quadrupole and dipole elements, parameterized as a linear integrable Hamiltonian by Courant and Snyder. The operational limits of high intensity accelerators are often determined by collective instabilites within the beam. Nonlinear elements may be added to supress certian forms of these collective effects, but restrict the range of stable trajectories. These shortcomings motivate extending to a nonlinear integrable which can share the benifits of supressing collective instabilites without limiting the stable trajectories. Danilov and Nagaitsev proposed a novel nonlinear integrable system with elliptical potentials which could be implemented as magnetic elements in an accelerator. \\ Such a system has been implemented for verification in the integrable optics test accelerator (IOTA), a small storage ring constructed for beam dynamics studies. Electron studies in IOTA treat the low-emittance beam as a macroparticle for a granular probe of the dynamics. Turn by turn measurements of the kicked beam responses are used for phase space reconstruction and analysis of the dynamics. Amplitude dependent detuning, a core figure of merit for suppressing instabilities, are directly measured and compared with high fidelity particle tracking simulations. Measuremnets of circulating beam with strong nonlinearities allow direct measurments of the topology of the nonlinear potential, and lifetime measurments in regions where nonlinear focusing terms dominate.}

%\makecopyrightpage

%\tableofcontents

\mainmatter

%\chapter{Itroduction} 

\section{Background}
Charged particle accelerators have been productive experimental tools for fundamental physics experiments, from rutherfords use of natural acceleration of alpha particles for the discovery of the nucleus to the contemperary multinational physics collaborations centered at the Large Hadron Collider. "Maybe add other physics discoveries" In addition to their useful nature as tools, particle accelerators provide an interesting system to study dynamics in a controlled environment. These same dynamical studies then have direct impacts on the construction of new machines for fundamental studies. As the experiments drive increasing energy and power demands on the beams, careful control of losses becomes more important.

To tackle the challenge of controlling energetic subatomic particles we first have to choose our tools. Some of the earliest "accelerator" experiments relied on energetic decay products, leveraging the weak nuclear force. This is quickly limited by the characteristic energies of these decays. Gravity can be considered and easily dismissed as weak and effectively fixed on the earth's surface. Bulk material interactions can be used to affect the path of particles, but they cannot accelerate and tend to cause significant losses. The default choice has been electromagnetic fields, dating back to early cathode ray experiments. These fields are easy to produce and control and can both steer and provide energy to charged particles. Naturally this restricts our ability to work with neutral particle beams, but this can be typically overcome with a charged primary beam to produce a neutral secondary beam as is the case in succesful neutron spallation and neutrino beam facilities.

Scaling the electrostatic fields from our early cathode ray tubes are a straightforward starting point for accelerating particles. A large voltage applied across carefully shaped electrodes can simultaneously accelerate and focus a beam. This was the guiding principle for early Van De Graf machines like the Westinghouse atom smashe and the Cockroft-Walton style proton sources popularly used in the mid 20th century. This approach still finds use in pellotrons and "tandem" accelerators for low energy nuclear experimentation. However, the practical limits of breakdown gradients quickly limit the energy from such devices. Using an oscillating electric field is a practical approach to bypassing this limitation in two ways. First, the gradients of oscillaitng fields may be much higher than static fields. Second, by selectively timing the particles to be accelerated, the particle can gain energy from the gradient at the frequency of the oscillation. There are a few ways this is currently accomplished. The first is simply shielding the particle from a travelling electromagnetic wave, as is done in the Alverez or drift tube linear accelerator (linac). 

So far the problem of imparting energy to the particles has been given precendence, but naturally the particles must also be steered and focused for useful purposes. Here we can consider magnetic fields as well as electric. While not possible to be used for energy gain, magnetic fields are the preferred option for steering and control. While electrostatic fields are used in some low energy sections, the proportional scaling of the force with momentum from a magnetic field pay divedends as beam energy increases. 

%\chapter{Integrable Optics Test Accelerator} \label{chap:iota}

\section{Facility Overview}

The Integrable Optics Test Accelerator (IOTA) was constructed at the Fermi National Accelerator Laboratory (Fermilab) for dedicated beam studies, particularly implementations of nonlinear integrable optics (NIO). IOTA is housed in the Fermilab Accelerator Science and Technology (FAST) facility and is not connected to the main sequence of proton accelerators at Fermilab. The location of the FAST facility on the Fermilab campus is marked in figure \ref{fig:FastMap}. In addition to IOTA, the FAST facility houses a superconducting electron linac, and a low energy proton injector.

\begin{figure}
	\centering
	\includegraphics[width=1\linewidth]{./chapter_2_figures/fermiBakedSmall.jpg}
	\caption{Location of FAST facility indicated by blue star at the end of a decommisioned fixed target beamline}
	\label{fig:FastMap}
\end{figure}


\section{FAST Electron Linac}
The electron linac started as a test bed for the ILC \cite{fermiILC,recordGradient}, and has since evolved to support a broad range of beam physics studies \cite{fastGreens,NEB}. Most importantly for the studies in this thesis, it is the source of the electron beams in IOTA. The general linac bunch parameters are given in table \ref{tab:linac}. The electron source is a gallium-arsenide photocathode inside a 1.3 GHz room temperature copper RF gun. The laser pulses for the photocathode are supplied from a supporting laser lab capable ofcareful tuning of the laser pulse shape and timing for granular control of the resulting electron bunch. The beam then travels through a low energy beam transport (LEBT) line consisting of two independent 1.3 GHz superconducting 9 cell "capture" cavities. The electon LEBT also contains a few experimental stations and an optional chicane for beam studies. At the end of the lebt the beam can be either seered to a low energy absorber or onward to the 1.3 GHz TESLA style cryomodule, which contains of 8 sequential 9 cell superconducting cavities. Beyond the cryomodule is a long transport line initially intended for further cryomodules, to a switchyard. Here the beam may be steered to further experiments, the high energy absorber, or injected into IOTA. Figure \ref{fig:linacLat} shows the lattice funcitons through the electron linac to the end of the IOTA injection septum. This long transport line presented some difficulties in tuning for injection. The $\beta$ functions neccesarially grow to large values which amplifies the impact of the quadrupole errors and granularities. While an accurate match into IOTA is irrelevant as the captured beam will damp to the equalibrium emittance, this touchy injection line was the cause of some headache in experimental operation, often limiting the injected current available.

\begin{table}
    \centering
    \begin{tabular}{lr}
    \toprule
    \textbf{FAST Linac Parameter} & \textbf{Design Value/Range}\\
    \midrule
    Beam energy to low energy absorber & 20-52 [MeV]\\
    Beam energy to IOTA/high energy absorber & 100-300 [MeV]\\
    Bunch Charge & $<\num{1e-5}$ - 3.2 [nC]\\
    Normalized Emittance (for 0.1 nC bunch) & 0.6 [mm-mrad]\\
    \bottomrule
    \end{tabular}
    \caption{FAST Electron Linac Parameters}
    \label{tab:linac}
\end{table}

\begin{figure}
	\centering
	\includegraphics[width=0.5\linewidth]{placeholder.pdf}
	\caption{FAST Electron Linac Lattice Parameters}
	\label{fig:linacLat}
\end{figure}

\section{IOTA Proton Injector}

The proton injector delivers 2.5 MeV proton beams to IOTA for space charge dominated studies. The hardware for the proton injection line is mostly repurposed from the Fermilab High Intensity Neutrino Source (HINS) \cite{webberHINS} project of the late 2000's, consisting of a Duoplasmatron proton source, a solenoidal low energy beam transport (LEBT) line, a radiofrequency quadrupole (RFQ), and a typical strong focusing medium energy beam transport (MEBT) line \cite{edstromIPI} to IOTA. The IPI line has been constructed and is undergoing comissioning at the time of the writing of this thesis.

The IOTA proton source is a 50 keV duoplasmatron. This particular source was characterized in the past \cite{tamThesis}. However, the current configuration of the source is not the state at the time of the original measurementnts. Some additional characterization was performed with an Allison type emittance scanner to verify approximate ion species fractions, but it was not possible to seperate proton emittance information. This will be evaluated with an allison scanner downstream of the RFQ.

The 325 MHz RFQ accelerates the beam to the final energy of 2.5 MeV. As the HINS linac was intended to be focused entirely with superconducting solenoids, the incoming and outgoing beams from the RFQ needed to be axially symmetric. This was accomplished through deliberately shaped electrodes for matching at the end of the otherwise typical RFQ vanes, and means that the matching into the RFQ needs round beams. 
The MEBT is mostly a straightforward strong focusing transfer line with two notable exceptions, a 325 MHz debunching cavity and an oversized dipole with strong edge focusing. The dipole was designed for operations with up to 3 GeV electrons for the superconducting linac at the facility. In order to properly steer the low rigidity protons into IOTA, the protons enter the dipole not at the pole face, but rather at the side of the pole with an effective entrance angle of 75$^\circ$. This combined with the relatively large bending angle of the dipole results in strong edge focusing. This strongly limits the degrees of freedom in matching the MEBT to IOTA, and means there are not free knobs for tuning injection into IOTA. In practice, the plan is to add small permanent magnet quadrupoles to the end of the MEBT to match the desired input distribution. The lattice functions for the IPI MEBT are given in \ref{fig:mebtLattice}, note the sharp kinks at the edges of the longest dipole, this is the effect of the strong edge focusing effects.

\begin{figure}
	\centering
	\includegraphics[width=1\linewidth]{./chapter_2_figures/adjustedMebt.png}
	\caption{IPI MEBT lattice functions}
	\label{fig:mebtLattice}
\end{figure}

\section{IOTA Design}
IOTA shows significant design influence from the VEPP-2000 collider at the Budker Institute for Nuclear Physics \cite{vepp-2000}. IOTA is of a stretched octangonal layout with four thirty degree and four sixty degree primary dipoles forming the fundamental geometry. IOTA originally was intended to be a regular octogon, but was eventualy stretched to better fill the availible lab space and provide a longer insert region for the optical stochastic cooling program. IOTA is broadly mirror symmetric across the streched sides, which informs the element naming scheme outlined in \ref{fig:iotaNames}. The dipole parameters are outlined in \ref{tab:dipole}. Primary focusing is provided by 39 quadrupoles of two types. High current quadrupoles are of a AMPS design from the Dubna laboratory, and the rest of ring is filled out with comercially available quadrupoles purchesed from Radiabeam. All quadrupoles are individually powered for flexible tuning of the lattice. The quadrupole parameters are outlined in table \ref{tab:quad}. IOTA has 20 pansofsky \cite{panofsky} style combined function correctors for local correction of the closed orbit and skew quadrupole terms. Table \ref{tab:corr} gives the important corrector parameters. For longintudinal focusing and making up for synchrotron radiation losses, IOTA has a single ferrite loaded quarter wave RF cavity operating at "". Injection into IOTA is facilitated by a horizontally bending Lambertson type magnetic septum \cite{lambertsonPatent???}, and a vertical travelling wave stripline kicker \cite{antipovKicker}. Additionally, there are two vertical correctors intended for injection bump manipulation, but are practically simply implemented as additional corrector knobs. In additon to the linear elements, IOTA has 12 sextupoles of three types. There are 4 "prototype" type sextupoles constructed in house, 6 "long" type sextupoles manufactured by Elytts to the same magnetic parameters as the "prototypes" and 2 "short" sextupoles which aim for the same magnetic properties with shorter poles to fit in a tight spot in the lattice. Sextupole parameters are given in \ref{tab:sext}. The basic IOTA geometry has three experimental insert locations. For the experiments described in this thesis, all three were filled. The first is the NIO insert described in detail in \ref{sec:nioDesign} in the BR straight. The BL straight housed a string of octupoles which satisfy one invariant of motion like the NIO system and are consider quasi-integrable. The octupole program lies outside of the scope of this thesis except for the magnetic alignment covered in chapter \ref{chap:magnets}. The final insert in DR was a permanent magnet undulator used for the CLARA experiment on electron radiation \cite{clara}. 

The IOTA beam diagonostics consisted of 21 button style beam position monitors, a wall current monitor, a direct current current transformer (DCCT), a photomultiplier tube for synchrotron radiation measurements, and five cameras for observing the synchrotron radiaiton. The BPM's were of two configurations, 20 nominal and one larger aperture BPM for increased admittance near the injection location. The BPMs sampled at 32 times the revolution frequency of the machine and applied a linear fit to the difference over the sum of the individual button digitized signals as described in \cite{linearBPM}. After obtaining the position from the button signals, a seventh order two dimensional polynomial was applied to the measured position to account for the nonlinear response of the BPM. The factors for the polynomial mapping were obtained based on pulsed wire data in a BPM. The BPMs could provide turn by turn data for "7000" turns, and 1000 turn averaged closed orbit data. Additionally 1000 turn raw button signals were availiable for a selectable single BPM in a special diagnosis mode. The IOTA wall current monitor was used for bunch length and synchrotron frequency measurements. 

\section{IOTA Lattice}
The dominant consideration for the design of the IOTA lattice is fulfilling the NIO insert requirements. These are:

\begin{enumerate}
	\item The beta functions in the insert must be symmetric between the x and y planes, which enforces a symmetric drift of matched phase advance which defines the overall rings fractional tune.
	\item The dispersion throughout the nonlinear insert must be zero.
\end{enumerate}

The basic transverse lattice requirements of the NIO system require 6 degrees of freedom to fully match. With the addition of dispersion suppression, this brings us to 7 degrees of freedom (assuming no coupling). IOTA is usually tuned to be mirror symmetric for convenience, which gives 20 quadrupole knobs, so there is available flexibility in the lattice for significant adjustments. In practice this was applied to keep the momentum compaction factor low, as a short bunch lenght was preferred for early experiments before this thesis.

\section{IOTA NIO Insert Design} \label{sec:nioDesign}
Once the insert drift parameters are fixed, the actual nonlinear insert magnets must be designed. To define the pole geometry, we introduce an additional function from \cite{mitchellComplex}. This is a complex representation of the vector and magnetic scalar potentials (not the relatavistic scalar potential), scaled by the rigidity.

\begin{equation} \label{eq:genF}
	F(z_c) = \frac{A_s}{B\rho} + i\frac{\Phi_m}{B\rho} = \frac{t c^2}{\beta(s)}\frac{z_c}{\sqrt{1-z_c^2}} \arcsin{(z_c)} \\ 
\end{equation}

We can can also expand this function about the origin, note that this expansion is only good inside the radius $x_c^2 + y_c^2 = 1$. This is of no practical concern as it only mandates that our expansion is good inside poles of the potential, where we are interested in the dynamics anyway.

\begin{equation} \label{eq:powF}
	F(z_c) = \frac{t c^2}{\beta(s)} \sum_{n=1}^{\infty} \frac{2^{2n-1}n!(n-1)!}{(2n)!} z_c^{2n}
\end{equation}

In a space with only the $A_s$ component of the vector potential, we know our magnetic field terms are given by Eq. \ref{eq:B_As}, with chain rule considered since our potential depends on the normalized coordinates.

\begin{equation} \label{eq:B_As}
\begin{split}
	B_x = &\frac{\partial A_s}{\partial y} = \frac{1}{c\sqrt{\beta(s)}}\frac{\partial A_s}{\partial y_c}\\
	B_y = -&\frac{\partial A_s}{\partial x} = -\frac{1}{c\sqrt{\beta(s)}}\frac{\partial A_s}{\partial x_c}
\end{split}
\end{equation}

To construct a Beth representation of our field we can combine our field terms in Eq. \ref{eq:cmBeth}.

\begin{equation} \label{eq:cmBeth}
	B_y + i B_x  = -\frac{\partial A_s}{\partial x} + i\frac{\partial A_s}{\partial y} = -\frac{1}{c\sqrt{\beta(s)}}\left(\frac{\partial A_s}{\partial x_c} - i\frac{\partial A_s}{\partial y_c}\right)
\end{equation}

A consequence of the Cauchy-Riemann equations yields Eq. \ref{eq:cauchy}

\begin{equation} \label{eq:cauchy}
	\frac{\partial F}{\partial z_c} = \left(\frac{\partial }{\partial x_c} - i\frac{\partial }{\partial y_c}\right) \frac{A_s}{B\rho}
\end{equation}

Substituting, we arrive at Eq. \ref{eq:BdF}, or the multipole expansion (in lab coordinates) Eq. \ref{eq:dnMult}.

\begin{equation} \label{eq:BdF}
	B_y + i B_x  = -\frac{B\rho}{c\sqrt{\beta(s)}}\frac{\partial F}{\partial z_c} = - \frac{t c B\rho}{\beta(s)^{3/2}} \left( \frac{z_c^2}{1 - z_c^2} + \frac{\arcsin{(z_c)}}{\left(1-z_c^2\right)^{3/2}}\right)
\end{equation}

\begin{equation} \label{eq:dnMult}
	B_y + i B_x = - \frac{t c^2 B\rho}{\beta(s)} \sum_{n=1}^{\infty} \frac{2^{2n-1}n!(n-1)!}{c^{2n}\beta(s)^n(2n-1)!} (x + i y)^{2n -1}
\end{equation}

There are two important takeaways from the multipole expansion, there are only even multipoles in the expansion (quadrupole, octupole, decapole, etc.), and the lowest multipole order is that of a quadrupole. So, the first order effect of the nonlinear insert can be treated in our linear dynamics for coarse tuning purposes and alignment. Figure \ref{fig:dnMultRatio} shows the deviation of the magnetic field of the multipole decomposition (Eq. \ref{eq:dnMult}), from the exact analytic form (Eq. \ref{eq:BdF} for increasing cutoff orders. Three points of comparison are taken, $(x=c/2,y=0)$, $(x=0,y=c/2)$, and $(x=c\sqrt{2}/2,y=c\sqrt{2}/2)$. 

\begin{figure}
	\centering
	\includegraphics[width=0.5\linewidth]{placeholder.pdf}
	\caption{Relative field magnitude deviation between exact form and expansion}
	\label{fig:dnMultRatio}
\end{figure}

The design of the nonlinear inserts use iron dominated magnets to shape the field, we can set to the scalar magnetic potential to a constant value and invert for the coordinates. In practice this is done numerically. The nonlinear insert used in these studies placed the pole at a scalar potential value of 0.5, in the $B\rho$ normalized units. Figure \ref{fig:dnMagPtCurve} shows the scalare magnetic potential gradient in the region where the multipole expansion holds, with the selected contour for the insert in red.

\begin{figure}
	\centering
	\includegraphics[width=0.8\linewidth]{./chapter_2_figures/DNscalarMagneticV1.pdf}
	\caption{Normalized magnetic potential of NIO system, contour used in IOTA magnetes is highlited in red. The region ner the singularity is masked to make the gradient more distinct.}
	\label{fig:dnMagPtCurve}
\end{figure}

Once the transverse profile has been defined the neccesary longitudinal scaling needs to be considered. The ideal DN potential smoothy scales with the beta function in the insert drift, but this is impractical to implement with magnets. For the real insert, a series of magnets approximatley integrate the potential. The basic implementation is to assume some number of thin kicks scaled by the hard edge equivalent lenght of the magnets. For the first version of the magnet, this was done with 18 magnets equally spaced along the lenght of the insert. Figure \ref{fig:dnLongKick} shows the effective potential for the ideal case and the 18 element piecewise approach. The poles must be shaped by the beta function at the given location to match the same contour in the magnetic potential. As a result the insert is composed of 9 unique pairs of magnets arranged symmetrically about the center. Figure \ref{fig:labPoles} shows the physical contours of the magnet families in the insert, the increase in size as the index increases from the center is clear. The nonlinear insert was designed in part to exploit the nonlinearities of the magnet at small amplitudes, so the $c$ parameter was chosen to be quite small; $c=0.009 [\sqrt{m}]$ (some literature claims a different c-parameter, but this value has been confirmed with the engineering drawings). This means that the singularity of the potential is at about 7 mm at the center of the insert, and neccesitates a carefully designed beam pipe to maximize the available aperture within the small poles. The mechanical, magnetic, and vacuum design and fabrication of the insert was performed by Radiabeam (in collaboration with the IOTA group) under a DOE small business innovation research (SBIR) grant \cite{radiabeamInsertReports}. The poles were manufactured using wire EDM.  



%\chapter{Simulation Studies} \label{chap:sims}

\section{Simulation Codes} \label{sec:simCodes}

There is a veritable zoo of commonly used codes for simulating the dynamics of charged particel accelerators, all with different focuses, operating assumptions, supported hardware, and dynamics models. A few different simulation softwares were used for simulation studies of IOTA and the NIO system, two main codes, and a few supplemental codes. The main lattice code for design and control of the linear optics was Six D Simulation (SixDSim), a java based application written and maintained by Alexsandr (Sasha) Romanov, the resident IOTA lattice expert. SixDSim performs lattice funciton calculations for periodic and channel mode accelerator systems in the full six dimensions, so first order dispersive and path lengthning effects are automatically considered. SixDsim provides general lattice construction and optimizing functionality with a graphical environment (a rarity in accelerator codes). In addition to offline lattice desing and manipulation, SixDsim has support for control system integration and was the main driving engine for the linear optimization of linear optics (LOCO) which defined the IOTA bare lattice for experiments. It also allowed for application of complicated custom knobs for fine tuning the lattice between time intensive LOCO iterations. As such, the SixDsim lattice model contained the best fit of the experimental IOTA lattice and was used for extraction of lattice funcitons used in calculations.

The other primary code used for IOTA simulations was Impact-X \cite{heubelNext}, a particle tracking code under active development with a particular focus on particle-in-cell space charge modelling and efficient operation on large scale computing environments. Impact-X was selected for two main reasons, the first is its implementation of the DN NIO lenses, and its advanced space charge models. The space charge simulation capibilites ended up being moot, as the proton program at IOTA was sufficeintly delayed to fall outside the scope of this dissertation. The implementation of the DN lenses in Impact-X is based on the complex representation formulated in \cite{mitchellComplex}, which has some benifits for numerical stability near the x-axis in configuration space. This is unlike other common tracking codes MAD-X and Xsuite which implement the original parameterization of the DN lens. Generally, Impact-X is a robust symplectic tracking code with modern Python scritping support and expanding physics cabibilities. The whole complement of tracking simulations presented were performed with Impact-X.

MAD-X is the closest the accelerator community has to a standard, and as such is the code in which the generic IOTA lattice is described. The cpymad wrapper for Mad-X was used as the benchmark for lattice functions in the tracking codes to evaluate proper lattice implementation, due to its straightforward implementation into python workflows. Xsuite is the new tracking code developed at CERN to replace and supplement the many legacy codes in use. For this work Xsuite was used for additional lattice benchmarking and as an educational tool. TRACK is a non-symplectic tracking code optimized for multi charge state acceleration of heavy ion beams in arbitrary external fields. The HINS RFQ re-used for the IOTA injector was designed in part with TRACK, so this most accurate physics model was used for that component of the proton injector simulation. Finally, a self programmed python tracking code was used for initial evaluation of invariant conservation in various nonlinear insert configurations. This was a valuable educational experience, but not a reccomended long term approach. All of these personally constructed lsimulations were re-performed with equivalent Impact-X lattices for better reproducibility and clarity.

\section{NIO Toy Simulations} \label{sec:dnSims}
The first simulations we will consider will be of the DN NIO system to demonstrate the underlying dynamics. The simulations are constructed in the drift-kick style, with the kick for the DN insert defined by the instantaneous change in momentum in a ruth-like formulation in equation \ref{eq:dnKick} \cite{mitchellComplex}. 

\begin{equation} 
\label{eq:dnKick}
\Delta p_x - i \Delta p_y = \Delta s \frac{t c}{\beta^{3/2}} \left( \frac{z_c}{1-z_c^2} + \frac{\arcsin{(z_c)}}{(1-z_c^2)^{3/2}} \right)
\end{equation}

The simplest insert configuration is a nonlinear insert drift with a thin matrix element providing the equal focusing. For ease of comparison, we match the IOTA NIO configuration with a 1.8 m drift with a phase advance of 0.3. This uiniquely constrains the lattice functions in the drift and sets the values in the focusing matrix. We will also add a limiting aperture of a circle with a radius of 0.95 times the lab frame location of the singularities in the DN potential. We are only interested in the dynamics between the singularities as this is the physically practical region. \cite{mitchellBiforcation} touches on the predicted dynamics outside of these constraints. We choose aperture of 0.95 of the limit to avoid numerical issues near the singularities. 

To best approximate the nominal smooth longitudinal scaling of the DN potential, we integrate the insert with a sequence of 1800 kicks, effectively subdividing the insert into millimeter-scale steps. The particles are initialized and monitored at the center of the insert. In the bare lattice this corresponds to the $\beta^*$ location where the $\alpha$ lattice functions are zero. This makes interpretation easy as the maximum amplitude of the oscillation shares the zero momentum condition. As we ramp the t-insert, the symmetry of the insert means the first order effect of the nonlinear magnets keeps the effective $\alpha$ zero at the center. The first simulation considered uses an inital grid of points which uniformly fills the central aperture in the transverse configuration space. Impact-X is a strictly 6-D code, so it is impossible to completely exclude the longitudinal dynamics, but the $ct$ and $p_t$ coordinates are uniformly zero, so any effects are the result of numerical error. The drifts in the nonlinear insert are strictly linear, and do not include chromatic effects. Figure \ref{fig:idealDNdetune} shows the resulting amplitude dependent detuning for this full configuration space range for a t-parameter of $t=-0.238$. The color indicates the relative inital coordinates, i.e. red points indicate only $y$ initial coordinates and blue the same for $x$. The tune was measured for each macroparticle with the NAFF algorighm \cite{laskarMeasureChaosNumerical1992,zisopoulosPZisoPyNAFF2023} and a first order Hann window \cite{harrisUseWindowsHarmonic1978a}. We see a very large tune footprint with a charactersitic "butterfly wing" profile. The largest detuning axis is dependent on the horizontal amplitude. There are some spurious features related to large horizontal excitations. We see a thin line of tunes crossing the main footprint from the lower "point", and a number of tunes on the $2Q_x + Q_y = 6$ resonance line. The interpretation of this effect becomes more obvious in the tune versus amplitude space.

\begin{figure}
	\centering
	\includegraphics[width=0.8\linewidth]{./chapter_5_figures/impactxIdealDNdetunet238.pdf}
	\caption{Amplitude dependent detuning in tune space for idealized NIO system at $t=-0.238$. Color indicates relative initial transverse coordinates.}
	\label{fig:idealDNdetune}
\end{figure}

Figure \ref{fig:idealDNampDetune} displays the tunes versus the initial amplitude. The amplitude is calculated as the initial emittance for the bare lattice, to easily compare between different simulations. All points are present in each plot, so the perpendicular ampligudes are also present for a given labeled amplitude axis. The color scaling for relative ratio of kick amplitude has been retained. In addition to the general tune dependence on different amplitudes, we have a clear indicator on these extra features. Both features are present at excitations leading to detuning near the vertical integer resonance. Noteably, since the system is fully integrable, all of these orbits remain stable. The interpretation of the tune becomes difficult in this region, as the TBT frequency approaches zero. The system is strongly nonlinear, so the assumption of a dominant frequency breaks down. The measured tunes for amplitudes predicted to detune beyond the also pose a practical challenge, standard signal proccessing methods are sensitive only to the fractional component of the tune, so we are insensitive to whether the tune "rebounds" or passes down by an integer. 

\begin{figure}
	\centering
	\includegraphics[width=1\linewidth]{./chapter_5_figures/impactxIdealDNtuneVampt238.pdf}
	\caption{Tune versus initial amplitude for idealized NIO system at $t=-0.238$. Color indicates relative initial transverse coordinates.}
	\label{fig:idealDNampDetune}
\end{figure}


The characteristic Poincar\`e sections of the DN NIO system are shown in Figure \ref{fig:idealDNpoincare} in fully normalized coordinates. The system exhibits these "Spirograph" like patterns near the circle which would be traced by the basic courant snyder system in the phase space. The most notable effect in the configuration space is the nonlinear couplig which results in this "Hourglass" shape that the particle traces. This has some impacts on admittance considered in section \ref{sec:DA}.

\begin{figure}
	\centering
	\includegraphics[width=1\linewidth]{./chapter_5_figures/impactxThinDNexamplePhase_t238.pdf}
	\caption{Poincare sections for a single particle of the horizontal and vertical phase spaces, and the configuration space for the idealized NIO system at $t=-0.238$}
	\label{fig:idealDNpoincare}
\end{figure}

We can gain some further insight by looking at the phase space and fourier specturm of a particle near the integer resonance. Figure \ref{fig:integerFourierSpectrum} shows the result of the FFT for such a particle. We see that there is a broader spectrum with many peaks, so our monotonal defenition of tune begins to break down somewhat.

\begin{figure}
	\centering
	\includegraphics[width=0.6\linewidth]{./chapter_5_figures/impactXthinDN_t238nearIntegerSpectrum.pdf}
	\caption{Fourier specturm of simulated particle near vertical integer resonance}
	\label{fig:integerFourierSpectrum}
\end{figure}

Figure \ref{fig:integerPhase} shows the phase space of this same particle. The motion becomes strongly nonlinear, especially vertically, where we are no longer orbiting around the original closed orbit.

\begin{figure}
	\centering
	\includegraphics[width=1\linewidth]{./chapter_5_figures/impactXthinDNexamplePhase_t238nearInteger.pdf}
	\caption{Phase space poincare sections for simulated macroparticle near vertical integer resonance}
	\label{fig:integerPhase}
\end{figure}

A final figure of merit to consider is the quality of the conservation of the analytically predicted invariants. Figure \ref{fig:toyInv} shows the calculted DN system Hamiltonian and the second invariant turn by turn from an example particle, normalized by the mean. In the ideal case this whould be flat. We see some slight jitter presumably from numerical noise and a slightly imperfect integration of the potential. This calculated quantity will be used for evaluating quality of other integrators and lattices moving forward as it is a direct indicator of the match of the system to the ideal DN NIO system.

\begin{figure}
	\centering
	\includegraphics[width=0.6\linewidth]{./chapter_5_figures/invariantConservation_t238_thinDN.pdf}
	\caption{Calculated analytic invariant quantites turn-by-turn from simulated particle in toy NIO system at $t=-0.238$}
	\label{fig:toyInv}
\end{figure}

We now need to consider more realistic nonlinear insert configurations. It is impractical to manufacture a magnet that smootly tapers with the required longitudinal scaling. To simulate the realistic IOTA insert, we need kicks which correspond to our 18 elements. This wil not perfectly match the neccesary integration of the potential, so we evaluate the quality of invariant conservation for this reduced number of elements. The following simulation consists of the IOTA style nonlinear insert with 18 equally spaced nonlinear thin kicks. We apply an aperture at the center of the insert consistent with the minimizing aperture in IOTA. Figure \ref{fig:v1idealInvariant} shows the resulting invariant conservation for the whole simulated grid of macroparticles. Here the color scale is the standard deviation of the invariants turn-by-turn and divided by the mean of the invariant quantity over the range, like in figure \ref{fig:toyInv}. The horizontal and vertical axes correspond to the simulated particles inital position in quadrant one, as the potential is symmetric about both axes. The aperture of IOTA is visible as the elipptical cutoff in the transverse dimensions. We see that the overall conservation quality has gone down, but our standard deviation is still below \num{1.1e-3}. In general, the quality of invariant conservation and approximation of the DN system improves for a larger number of integration steps or slices, though we are limited to what we can practically build.

\begin{figure}
	\centering
	\includegraphics[width=1\linewidth]{./chapter_5_figures/toy_equiSpace_t238_invariants.pdf}
	\caption{Invariant quantities conservation for the full IOTA aperture with 18 thin NIO kicks at $t=-0.238$}
	\label{fig:v1idealInvariant}
\end{figure}

The $t=-0.238$ is the nominal t-parameter studied further in this dissertation, but we are also interested in the quality of conservation in more nonlinear-dominated configurations. Identical simulations were performed with the 18 element configuraton for a t-parameter $t=-0.45$, which is almost at the vertical integer resonance. The conservation of both invariant quantites was below a fractional standard deviation of \num{2.5e-3} for this t-parameter.

The equal spacing configuration is the simplest apporoach to the system, but alternative configurations have been found have superior quality of integration. \cite{baturinHamiltonianPreservingNonlinear2022} proposes two alternative integration schemes, an equi-phase insert configuration, and an Yoshida-style \cite{yoshidaConstructionHigherOrder1990} higher order integrator. The equi-phase style integration can be straghtforwardly implemented as a new insert. Instead of placing elements equally spaced in lab frame positon, the elements are equally spaced in bare lattice phase advance. A simulation using 11 equi-phase elements instead of 18 at a t-parameter of $t=-0.238$ showed invariant conservation below a fractional standard deviation of \num{4e-3}. The upper limit on the conservation deviation is dominated by a single region in the configuration space near the horizontal aperture limit. The majority of the range has conservation better than \num{1.5e-3} like the 18 element equal space case. This is a significant advantage in practically construcing these inserts, as we can arrive at similar NIO system quality with fewer expensive magnets. This configuration is the basis for the second IOTA insert covered in appendix \ref{apx:dnv2}.

\section{IOTA Tracking Simulation Lattices}
For better fidelity simulations of experimental design and comparison, we need to include the full IOTA lattice. The physical dimensions the Impactx IOTA lattice used are based on the publically released version of the Mad-X lattice. The linear dynamics were benchmarked to this lattice configuration with simple gaussian beam simulations as Impact-X does not have direct first-order lattice calculation capibilites. The lattice was configured to accept quadrupole settings from SixDSim lattices. In all Impact-X simulations, the fully symmetric model of IOTA was used, matching the lattice in figure \ref{fig:inj6a}. This is without BPM offsets, dipole gradients, and the undulator corrections for easiest interpretability. 

Impact-X supports a number of higher order effects which we can optionally include in the simulations to verify their impacts on invariant conservation and stability. The base condition considered above uses the linearized versions of the linear elements, i.e. drifts, quads, and dipoles. With respect to longitudinal effects, this means that we are considering disperion but not chromaticity. To begin to consider chromaticity, we change the quadrupole model to a drift kick model which integrates the full nonlinearities present, and the drifts to the exact model which includes the full geometric nonlinear term in momentum. The combination of this drift model and thin kicks is a sufficient Hamiltonian splitting for  arbitrary transverse kicks, so adding nonlinear elements in this lattice properly inlcudes their chromatic effects as well. In this configuration we can evaluate the chromaticity of the lattice, by directly measuring the tune dependence on energy. We see a discrepancy in the natural chromaticity of the Impact-X lattice compared with SixDSim. Additionally, the impacts of the sextupole families based on the nominal calibrations fail to accurrately compensate the chromaticity, this likely stems from improper calibraiton of the sextupole current to strength. Regardless, based on the measured impacts of the sextupoles in Impact-X, we can fully compensate the natural chromaticity in the simulated lattice. For bunched beam simulations in the chromatic lattice we implement a thin linear model of the RF. The bunch length is small compared to the RF bucket, and synchrotron dynamics are not typically relevant compared to momentum deviation and simple stablity, so the linear approximation is sufficient. In addition to chromatic effects, we can introduce models of the expected residual nonlinearities in the bare lattice. The two elements considered in Impact-X are quadrupole fringe fields \cite{forestLeadingOrderHard1988} and the "geometric" nonlinearites from the short bending radii of the dipoles \cite{bruhwilerSymplecticPropagationMap}. In addition to the larger nonliner contributions full, nonlinear drift-kick models of the sextupoles can be implemented. We have one more configuration to consider, which is a first-order version of the NIO insert. This is used to evaluate the general lattice with only the first order impacts of the NIO insert. Since the insert is modeled with thin kicks, in Impact-X this is accomplished by simply replacing the DN kicks with thin quadrupole kicks, not the typical linearized thick quadrupole models used elsewhere in the ring. We now have many basic lattice configurations which will be swapped between in the upcoming chapters, the have been given nicknames and associated descriptions in table \ref{tab:iotaLats}. Sextupole strenght configurations are in general more flexible and treated seperately.

\begin{table}
    \centering
    \begin{tabular}{l|p{5.2in}}
    \toprule
    \textbf{Nickname} & \textbf{Lattice Description} \\
    \midrule
    toy & Thin, equal focusing matrix for linear matching. 1800 equally spaced thin kicks for NIO insert.\\
    \midrule
    linmin & Linearized IOTA model, linear thick quads, linear thick dipoles, linear drifts. Minimizing aperture matching that in real IOTA. 18 thin kicks for NIO insert matching physical insert, with linear drifts between.\\
    \midrule
    crom & Nonlinear drift-kick quadrupoles, exact nonlinear drifts, linear thick dipoles. Minimizing aperture matching that in real IOTA. Thin sextupoles may be used, treated seperately. Thin linear RF modle included for bunched beam simulations. 18 thin kicks for NIO insert matching physical insert, with exact nonlinear drifts between.\\
    \midrule
    nonlin & Nonlinear drift-kick quadrupoles with thin fringe field models, exact nonlinear drifts, exact nonlinear dipoles. Minimizing aperture matching that in real IOTA. Drift-kick sextupoles may be used, treated seperately. Thin linear RF modle included for bunched beam simulations. 18 thin kicks for NIO insert matching physical insert, with exact nonlinear drifts between.\\
    \midrule
    quadnio & Same matching lattice configuration as "nonlin", but 18 NIO kicks in insert are replaced with proportionatley scaled thin quadrupole kicks.\\
    \bottomrule
    \end{tabular}
    \caption{IOTA Simulation Lattice Configurations}
    \label{tab:iotaLats}
\end{table}

In addion to the linear lattice benchmarking, identical simulations using the "linmin" lattice to those carried out with the thin matrix yielded similiar invariant conservation levels as seen in Figure \ref{fig:v1idealInvariant}. We can look at the detuning range given by the realistic aperture in Figure \ref{fig:IOTAminapFullDetune} for $t=-0.238$ in "linmin". The overall detuning shape is the same, but the range is much smaller, especially from the reduction in horizontal aperture. The interpretation problem of the tunes near the integer resonance disappears in this realistic configuration, which simplifies the experimental analysis.

\begin{figure}
	\centering
	\includegraphics[width=0.8\linewidth]{./chapter_5_figures/impactxIOTAlinearDetune_t238.pdf}
	\caption{Amplitude dependent detuning for IOTA with  $t=-0.238$}
	\label{fig:IOTAminapFullDetune}
\end{figure}


\section{Energy Dependence Simulation}
We can evaluate the impacts of realistic momentum deviation on the system with controlled offsets. In the following simulations, we consider an initial distribution consisting of transverse particles which fill the available aperture at a select group of initial momentum offsets. We are primarily interested in the relative stability at these locations, so there is no RF applied, and the macroparticles are permitted to slip longitudinally as in a coasting beam. Before we commit to these simulations, we will verify that the grid in xy is a sufficient inital condition to sample the whole phase space. For the linear hamiltonian, this will uniformly cover the available phase space, but the full phase space of nonlinear systems is not neccesarially sampled by a uniform distribution in one phase space variable. We have some confidence, because the underlying system is integrable it is smoothly differentialble in phase space variables, so there should be no isolated regions in phase space. This assumption breaks down with perturbations to the system in the form of additional nonlinearities. To verify, a simulation in the "crom" with 4-D initial conditions was performed. The base structure consisted of a $x$-$y$ grid, and for each point in configuration space a smaller range of $p_x,p_y$ points was added. The energy offsets were also included to evaluate their impacts. Figure \ref{fig:fullPhaseDNH} shows the fractional standard deviation in the DN hamiltonian plotted against the initial positions in phase space. The plotted points have no momentum deviation. All simulated particles are plotted, so naturally, some are obscured by later points, here they have been sorted by conservation so the "worst" points rise to the top. 

\begin{figure}
	\centering
	\includegraphics[width=1\linewidth]{./chapter_5_figures/phaseVolCorrCromlatInj6a_t238.png}
	\caption{DN Hamiltonian fractional standard deviation against transverse phase space variables}
	\label{fig:fullPhaseDNH}
\end{figure}

We see that generally, conservation is reasonable with the nonlinear contributions from chromatic drifts and quads. More importantly, we can see that there is no significant correlation on invariant conservation agains initial momentum. The only point of concern is the cluster of unusual poor conservation around $x=3,y=2$. This picture becomes more clear if we instead plot the conservation with respect to the maximum amplitude as calculated by the effective emittances. Figure \ref{fig:invVenergy4D} shows the fractional invariant conservation for both DN invariants, Hamiltonian on the left and second invariant on the right. Here the color scales are unified, if invariant standard deviation is below \num{5e-3} it is colored dark blue, and if over \num{5e-2} orange. The horizontal and vertical coordinates are the maximum amplitudes at the IOTA minimum aperture calculated by using the first order effect of the nonlinear insert to calculate the effective emittances. The third plot from the top on the left corresponds to the same points, and we see that this region of poorer conservation in \ref{fig:fullPhaseDNH} corresponds clearly to a line in the reduced quality conservation in combined amplitude, presumably stemming from a nonlinear resonance.

\begin{figure}
	\centering
	\includegraphics[width=0.6\linewidth]{./chapter_5_figures/t238_inj6aQuads_cromLat_invariantVenergy4DMask.png}
	\caption{DN Invariants fractional standard deviation against maximum amplitudes calibrated by effective courant snyder emittances and initial momentum offsets}
	\label{fig:invVenergy4D}
\end{figure}

We can further compare this conservation picture with a similar plot of the invariant conservation for only intial position seeds in \ref{fig:invVenergySplit}. All of the same features are present at a higher resolution for many fewer simulated particles due to the reduced dimensionality.

\begin{figure}
	\centering
	\includegraphics[width=0.1\linewidth]{./placeholder.pdf}
	\caption{DN Invariants conservation for flat intial positions with all momentum $p_x,p_y$ zero.}
	\label{fig:invVenergySplit}
\end{figure}

We now want to evaluate the impacts of chromatic compensation with sextupoles on the dynamic aperture and invariant conservation. This is the generic approach to matching the chromaticities and maximizing decoherence times for the bare lattice in experiment. This should nominally compensate for the energy dependence of the invariant conservation seen in figure \ref{fig:invVenergySplit}. Figure \ref{fig:invVenergySext} shows the resulting conservation plots. We see that conservation is signficantly impacted, the threshold for the orange points has been raised to a fractional standard deviation of \num{5e-1} for reasonable comparison. Clearly this means that the NIO system is not well represented here. Additionally, the sextupole effects introduce a significant reduction in the dynamic aperture, especially in regioins detuning to the sextupole resonant lines.

\begin{figure}
	\centering
	\includegraphics[width=0.1\linewidth]{./placeholder.pdf}
	\caption{DN Invariants conservation for flat intial positions with all momentum $p_x,p_y$ zero.}
	\label{fig:invVenergySext}
\end{figure}

We can also simulate the impacts of expected nonlinearities in our lattice with the "nonlin" lattice configuration. Here we do not include sextupole contributions. This condition is milder than the sextupole impacts, we can revert to the orange upper limit at \num{5e-2}. Generally, outside of the impacted upper regions


Finally, we can consider the expected nonlinearities and the sextupole compensation. We need to increase the orange threshold to \num{5e-1} again. invariant conservation is poor, and dynamic aperture is significantly reduced, though the main impacts seem to stem from the sextupole terms. It is difficult to evaluate the invariant conservation dependence on the energy offset.


\section{Bunch Simulations}
The IOTA electron experiments leverage the fact the electron beam is small and can act as a macroparticle probe of the dynamics. Some simulations were performed to verify that this assumption holds in the strongly amplitude dependent NIO system. 

\section{Theoretical Lattice Improvement Simulations}
Based on experimental results, a number of potential improvements to the bare lattice configuration were considered in simulation. The simplest impact evaluated was to minimize the emittance of the electron beam. This has the effect of making the beam more point-like, and most importantly reduces its tune footprint. A reduced footpring means extended coherenet centroid oscillations and improved sensitivity for all major turn-by-turn measurments. The main contributor to emittance is the dispersion in the 60 degree dipoles. For the NIO lattice, the dispersion and its derivative must be zero at the end of these dipoles on the face near the insert to match the NIO condition. In the dipole the derivitive of the dispersion monotonically increases, so the simplest approach to reduce these integrals in the 60 degree dipoles is to configure them as a so called double bend achromat \cite[pg.133]{leeAcceleratorPhysics2018}. An example lattice which retains the core NIO requirements with a DBA like structure in the 60 degree dipoles is presented in Figure \ref{fig:dbaSixdsim}. This lattice results in over an order of magnitude reduction in equalibrium emittance in the horizontal plane, from \num{1.67e-5} to \num{1.01e-6} [cm-rad] as calculated by SixDsim using the radiation integrals \cite[pg.438]{leeAcceleratorPhysics2018}.

\begin{figure}
	\centering
	\includegraphics[width=1\linewidth]{./chapter_5_figures/dbaLatticeCropped.png}
	\caption{SixDsim lattice plot of the lower emittance DBA-style IOTA configuration.}
	\label{fig:dbaSixdsim}
\end{figure}

The resulting detuning impacts on a simulated TBT response can be seen in Figure \ref{fig:lowEmitDetune}. The tune footprints for an example kick amplitude of beams at the equalibrium emmittance for the generic NIO and low-emittance lattices are plotted in blue and green respectively. 

\begin{figure}
	\centering
	\includegraphics[width=0.8\linewidth]{./chapter_5_figures/bunchKickEmitOptTunest238.png}
	\caption{Single bunch kicked beam tune footprints before and after lattice emittance minimization.}
	\label{fig:lowEmitDetune}
\end{figure}

The results on the decoherence as measured by the mean of the transverse coordinates are given in Figure \ref{fig:lowEmitDecohere}. We see about a factor of two improvement on the vertical decoherence, a relavant factor as this is often the limiting number in the tune and phase space reconstruction. 

\begin{figure}
	\centering
	\includegraphics[width=1\linewidth]{./chapter_5_figures/bunchKickCentroidEmitOp_t238.pdf}
	\caption{Decoherence profiles for kicked beam simulations before and after lattice emittance minimization.}
	\label{fig:lowEmitDecohere}
\end{figure}

This lattice represents only the first pass on the optimization, but is readily applicable for the IOTA lattice and within the bounds of the current quadrupoles. In addition to the emittance minimization, tuning of the natural chromaticities of IOTA was considered. The main consideration for NIO lattice operation is matched chromaticities in the horizontal and vertical planes. The natural chromaticites in IOTA are already near each other. A lattice was optimized to bring these chromaticities to match, though the symmetry of IOTA had to be broken. This was accomplished with a general optimization using the full range of IOTA quadrupoles, there is not a simple linearly scaling knob to adjust the natural chromaticities in this way. Due to the imperfect matching between the SixDSim and Impact-X calibrations, the quality of the NIO condition degraded somewhat. The resulting invariant conservation versus energy spread is given in Figure \ref{fig:chromMatchInv}. We can see that the overall conservation in the zero energy condition is slightly comprimised, likely due to this imperfect bare lattice. But, more importantly, the invariant conservation for off momentum particles is improved, especially for higher amplitudes.

\begin{figure}
	\centering
	\includegraphics[width=0.1\linewidth]{./placeholder.pdf}
	\caption{Invariant conservation for lattice with approximately matched natural chromaticities}
	\label{fig:chromMatchInv}
\end{figure}

\section{IOTA Proton Injector Simulations}
While electron experiments are the primary focus, the IOTA proton injector line was being formalized and constructed during the course of this work. Some simulations were undertaken for evaluating the impact of space charge effects in the proton injector beamline. 


%\chapter{Experimental Analysis} \label{chap:analysis}


\section{Tune Measurements} \label{sec:tune}
Tune measurments provide long term coherent information on the dynamics and the strong dependence on amplitude is the chief experimental indicator of nonlinearities in the focusing. While decoherence limits the available turns for measurments, the variations in amplitude do not alter the fundamental frequency, which gives us a longer coherent signal. The tune was previously defined only for the linear system in the context of phase advance. The definition of the tune is more nebulous then for a strongly nonlinear system, as the motion is neccesarially anharmonic. For nonlinear dynamical systems, the poincare rotation number is a related quantity \cite{nagaitsevBetatron}, but difficult to relate to measurable quantities. For the experimental measurments, we interpret the dominant frequency of the transverse oscillation spectrum as the tune. 

Different tune measurment algorigthim were considered for the particular limitations of the available data. The simplest approach is simply taking the Fourier transform of turn by turn (TBT) BPM data and picking the peak frequency. As we have discrete sampling of the motion, a discrete Fourier transform is a simple first approach. Practically, the fast fourier transform (FFT) is used as the default discrete transform. There are a number of methods to improve the resolution of the frequency measurements. In principle a windowing function can be applied to improve resolution of the peak frequency at the cost of supressing sidebands. However, for the IOTA data, there are two important considerations. A window suppresses the amplitude of a fraction of the sample window. In the case of the short coherent measurments, the reduction in available signal was more detrimental than the advantages of the window. And for low signal to noise ratios the benefits of a window disappear. For example, the common Hann window does not provide any benefits for signal to noise ratios less than \num{1e3} \cite{bartoliniAlgorithms}, well above those seen in IOTA. To bypass some of the limitations on the disctrete sample resolution of the FFT, Jacobsen interpolation \cite{jacobesnLocla} applies a quadratic interpolation to the FFT peak and its nearest neighbors for finer peak resolution. The implementation used is based on that from the PyLHC \cite{cernomc} analysis library. The other approach considered is the Numerical Analysis of Fundamental Frequencies (NAFF) proposed by Lasker for the evaluation of long term stability of planetary systems \cite{laskar}. NAFF starts with a guess from the FFT spectrum, and seeks to iteratively optimize the magnitude of a single fourier transform as in equation \ref{eq:naff}. In principle this can be used to extract successive harmonics of the motion, but we are interested only in the dominant frequancy, so typically only one term is considered. The implementation of NAFF used in this analysis is PyNAFF \cite{zisopolusPZ}, this uses hardys method (Eq. \ref{eq:hardy} \cite[p.151]{whitakerCalculs}) for the numerical integration as opposed to a simple riemann summ. In addition to the offline methods, live tune measurments on circulating beam used a least squares approach, where an assumed functional form with free parameters for tune, phase, coupling, and decoherence times, is fit to the data.

\begin{equation}
	\Psi(T) = \frac{1}{2T}\int_{-T}^{T}f(t)e^{-i\omega t}dt
	\label{eq:naff}
\end{equation}

\begin{equation}
	\int_{x_{n}}^{x_{n+6}}f(x)dx \approx \frac{\Delta x}{100}\left(28(f(x_n) + f(x_{n+6})) + 162(f(x_{n+1}) + f(x_{n+5})) + 220 f(x_{n+3}) \right)
	\label{eq:hardy}
\end{equation}

To evaluate the methods, a comparison was made with both a synthetic signal and the simulated bunch centroid signal from an Impact-X simulation. Figure \ref{fig:synKicks} shows the evaluation signals, the amplitude does not impact the tune evaluation. The synthetic signal consisted of a convolution of a single harmonic base signal, a logistic function for decoherence and a gaussian noise distribution with a RMS of ten percent of the signal amplitude. 

\begin{figure}
	\centering
	\includegraphics[width=1\linewidth]{./chapter_3_figures/syntheticAndSimulationKicks.pdf}
	\caption{Evaluation signals, fully synthetic and simulated}
	\label{fig:synKicks}
\end{figure}

The first evaluation was on the convergence of the various methods. Figure \ref{fig:baseConv} shows the convergence of tune measurments for the various considered algorighms with increasing length of the sample window. The limitations of the FFT resolution from the available signal are clear, but they inform the initial values for the interpolation and the NAFF. The vertical simulation signals are considered here as they exhibit faster decoherence. The relatively long coherence time of the horizontal signal means that convergence rate is less important, and we are interested in best performance for very fast decoherence.

\begin{figure}
	\centering
	\includegraphics[width=1\linewidth]{./chapter_3_figures/basicConvergence.pdf}
	\caption{Convergence of tune measurments with lenght of signal window}
	\label{fig:baseConv}
\end{figure}

The available resolution of the FFT can be improved by padding the input signal with zero values. This is the same effect as increasing the lenght of the signal and applying a rectangular window the length of the original signal. Figure \ref{fig:padConv} shows the convergence of the faster methods with and without padding. Here all signals were padded to an equal signal lenght of 256 turns. We see that the convergence of the padded FFT is improved, but the windowing effect introduces a systematic, limited offset of the tune determined by the lenght of the padding. The Jacobsen interpolation with a padded signal also suffers a similar effect, converging on a systematically offset tune.

\begin{figure}
	\centering
	\includegraphics[width=1\linewidth]{./chapter_3_figures/paddedConvergence.pdf}
	\caption{Convergence of tune measuremnts with zero padding}
	\label{fig:padConv}
\end{figure}

We then have three approaches which reliably converge before the 100 turn limit of our sample plots, Jacobsen interpolation wihtout padding, and NAFF with and without padding. In evaluating the quality of the convergence, the noise seed on the simulated signal was found to have a significant effect on these methods. The convergence of the tune measurement was evaluated for the same underlying synthetic signal and decoherence as before but with one thousand different noise seeds for the random noise. The synthetic signal was used as there is a garuntee of a single underlying frequency unlike the simulation signal. The ensemble properties of the convergence are plotted in Figure \ref{fig:noiseConv}. The left plot shows the mean of the absolute difference of the measured tune from the nominal, $|Q_{meas} - Q_{nominal}|$ and the right plot is the standard deviation of the measured tunes for each window lenght. We see that the average convergance is about the same, with a slight prefereance for Jacobsen interpolation. The padded NAFF measurements have slightly lower variation.


\begin{figure}
	\centering
	\includegraphics[width=1\linewidth]{./chapter_3_figures/noiseSeedConvergence.pdf}
	\caption{Convergence of tune measuremnts with zero padding}
	\label{fig:noiseConv}
\end{figure}

There is an additional consideration, we can evaluate the accuracy of the measurments while adjusting the inital phase. There is a periodic variation of the measured tune for this change in initial.  This neccesarally means sampling a lower amplitude region of the signal as you go along the decoherence. With the padded signal we can also do a similar transfomation. By rolling the padding to the front as well as behind with the same overall signal lenght, different phases given by the resoltion of the binning. Figure \ref{fig:baseRoll} shows the change in the tune measurments for varying the initial turn of a sixty turn sample window and "rolling" the padded signal. Note that the first value of the NAFF padded and the NAFF rolled circled in black are exactly the same as they are evaluating the same signal. This comparison is not apples-to-apples but it does imply a benefit of the NAFF padded evaluation. For a given signal lenght the uncertianty due to the initial phase is regular and can be evaluated without sampling different, presumamby lower amplitude areas of the original signal.

\begin{figure}
	\centering
	\includegraphics[width=0.7\linewidth]{./chapter_3_figures/slideTuneComparison.pdf}
	\caption{Variaiton of tune measurments with inital phase}
	\label{fig:baseRoll}
\end{figure}

The different tune measurement algorithms have different characteristic statistical uncertianties that scale with the availible sample window. Without interpolation, the uncertianty in the FFT lines goes as $\frac{1}{N}$ where N is sample points, in our case available turns. Both the unwindowed NAFF and interpolation approaches have analytic uncertianties that go as $\frac{1}{N^2}$ \cite{zisopolousRefined,bartoliniTune}. The magnitude of the variation due to initial phase was added to the uncertianty of the tune measurements. We then have the following steps for a single signal tune measurement:

\begin{enumerate}
	\item{Zero pad signal to predetermined length}
	\item{Evaluate the tune with NAFF}
	\item{"Roll" the signal to shift inital phase}
	\item{Evaluate rolled signal with NAFF}
	\item{Repeat roll and evaluation a predetermined number of times, and estimate initial phase uncertianty from variance in rolled tune measurments}
\end{enumerate}

This algorightm is applied to the real TBT bpm sample (Figure. \ref{fig:realEvalSig}) in  Figure \ref{fig:measConvRoll}.

\begin{figure}
	\centering \includegraphics[width=1\linewidth]{./chapter_3_figures/realTuneEvalSignal.pdf}
	\caption{Example kick for tune evaluation, BPM B1R signal}
	\label{fig:realEvalSig}
\end{figure}


\begin{figure}
	\centering \includegraphics[width=1\linewidth]{./chapter_3_figures/realTuneConvergence.pdf}
	\caption{Tune measurment convergence for measured TBT data, with estimated uncertianty}
	\label{fig:measConvRoll}
\end{figure}


The above comparisons assume a single signal, but multiple BPM signals are available and we would like to combine them, there were two approaches considered. The first is a BPM stacking approach, where a quasi-periodic signal is constructed \cite{zisopolousRefined} by stacking sequential BPMs. This approach did not yield good results for the experimental IOTA data.

The other considered approach to investigate multiple BPM response was to select the dominat components of a PCA decomposition of all combined BPM signals. PCA was selected as the scikit-learn implementation of PCA was significantly more performant than the numpy implementation of SVD, with the same resulting singular values. Figure \ref{fig:pcaComps} shows the matrix resulting from the convolution of the BPM-specific terms in the PCA with the singular values of a single kick, the same kick as shown in \ref{fig:realEvalSig}. Each column corresponds to a BPM, with the center line indicating the split between the horizontal and vertical BPM labels. 

\begin{figure}
	\centering
	\includegraphics[width=0.7\linewidth]{./chapter_3_figures/t238_singleKickPCA.pdf}
	\caption{PCA decomposition of TBT signals }
	\label{fig:pcaComps}
\end{figure}

The rows indicate different temporal signals. To provide some insight, in the case of simple harmonic motion we would expect four terms, two components for each phase in each plane. Each BPM signal is then composed of the relevant fraction of these phase components with scaling proporitonal to the beta function at that location. Figure \ref{fig:pcaTemps} shows the first four temporal components of the same decomposition. 

\begin{figure}
	\centering
	\includegraphics[width=1\linewidth]{./chapter_3_figures/pcaTemporalModes.pdf}
	\caption{First four PCA temporal modes from decomposition above}
	\label{fig:pcaTemps}
\end{figure}

We can then discriminate on the direction of a tune component by looking at the relative magnitudes of the temporal components in the horizontal and vertical BPMs. A tune measurement technique, NAFF in this case, can then be applied to these individual components. This has the slight drawback of making phase measurments of individual BPMs more difficult, but this is not typically considered the the analysis anyways. 

Another effect on tune measurements is resonant capture, where a small fraction of the beam becomes trapped in a resonant condition and continues to oscillate at the characteristic frequency of the resonance. As the rest of the beam quickly decoheres due to tune spread, this dominates tune measurements with long sample windows. This can be seen in the spectrogram with short vertical decoherence with a long faint line. Figure \ref{fig:resCapReal} shows the profile for a kick demonstrating resonant capture. The vertical decoherence is very fast, about 20 turns. However, a small portion of the beam continues to coherently oscillate as can be seen in the spectogram in Figure \ref{fig:specResCap}. This spectogram has a window lenght of 60 turns. The dominant peak in the first window is with the decohering main tune line, but we see a faint line with one bin lower tune and long coherence right on the third order difference line $Q_x-2Q_y = 0$.

\begin{figure}
	\centering
	\includegraphics[width=1\linewidth]{./chapter_3_figures/resonantCaptureSignal.pdf}
	\caption{Example resonant capture kick}
	\label{fig:resCapReal}
\end{figure}


\begin{figure}
	\centering
	\includegraphics[width=1\linewidth]{./chapter_3_figures/resonantCaptureSpectogramB1R_238_10_19_kick26.png}
	\caption{Resonant capture spectogram}
	\label{fig:specResCap}
\end{figure}

This resonant capture effect means we prefer to keep the tune measurment windows as short as gives reasonable tune resolution. This effect has been demonstrated in simulation, with kicked bunch measurements, and can be seen in single-electron measurments.

\section{Nonlinear Insert Calibration} \label{sec:nioCal}
The nonlinear insert had to be calibrated and aligned to best match the nominal potential. This was done with beam based measurments. Recalling from section \ref{sec:dnMag} that the lowest order component of the nonliner insert is a quadrupole, and that the proper implementation of the DN NIO system requires consistent longitudinal scaling to properly match the potential, we treat the individual DN magnets as quadrupoles for small amplitudes. LOCO was applied to center the individual elements, the measured closed orbit offsets after centering \textit{assuming quadrupole terms} are presented in \ref{fig:dnOffset}. 

\begin{figure}
	\centering
	\includegraphics[width=0.7\linewidth]{./chapter_3_figures/dnOrbitOffsets.pdf}
	\caption{Closed orbit offests in DN magnet after manual alignment, assuming quadrupole terms}
	\label{fig:dnOffset}
\end{figure}


Each magnet was energized individually and a small amplitude kick was applied to the beam to measure the tune shift. This was done for multiple current setpoints to fit a tune shift vs current.
The individual scaling of the nonlinear t-parameter to current was calibrated for each magnet using the tune shift due to a quadrupole error in the lattice, (eq. \ref{eq:quadErrShift}) set equal to the quadrupole term of the DN potential (eq. \ref{eq:dnQuadTerm}), and solved for t.

\begin{equation}
    	\Delta Q_{x} = \pm \frac{1}{4\pi}\int{\beta_x(s) \frac{\Delta B_2}{B\rho} ds}
    	\label{eq:quadErrShift}
\end{equation}

\begin{equation}
	\Delta B_2 = \frac{-2B\rho \Delta t}{\beta^2(s)}
	\label{eq:dnQuadTerm}
\end{equation}

The resulting data gives a relative current scaling profile for the overall insert Fig. \ref{fig:dnIscaling}. This calibration assumes good accuracy of the beta functions in the nonlinear insert at the time of the calibration.

\begin{figure}
    \centering
    \includegraphics[width=0.7\linewidth]{./chapter_3_figures/2023-05-22dnScalings.pdf}
    \caption{Relative DN insert current scaling profile compared to calculated ideal values}
    \label{fig:dnIscaling}
\end{figure}

This approach gives good relative scaling of the magnets, but we want to additionally evaluate the calibration of the insert as a whole. The tune shift assuming only the quadrupole terms of the DN element is given in Eq. (\ref{eq:dnDetune}) \cite{nagaitsevNonlinearOptics}.

\begin{equation}
	\begin{split}
	Q_{x} = Q_{o}\sqrt{1+2t} \\
        Q_{y} = Q_{o}\sqrt{1-2t}
	\end{split}
	\label{eq:dnDetune}
\end{equation}

The tune shift of the insert was measured by adjusting the t-parameter of the entire DN insert, according to the previously calibrated current ratios and kicking the beam to small amplitudes to measure the tune. The measured tunes are plotted with the quadrupolar detuning in Figure \ref{fig:dnDetuning}. The impact of the sextupoles used to compensate chromaticity can be seen inas the dominant effect on the tune near the third order resonances (red lines on tune diagram).

\begin{figure}
    \centering
    \includegraphics[width=0.8\linewidth]{./chapter_3_figures/tuneSpaceDnShift.pdf}
    \caption{Measured nonlinear insert tune shift detuning}
    \label{fig:dnDetuning}
\end{figure}

The ratio of the tune shift between the planes is good indicator of proper implementation of the longitudinal scaling of the potenetial and beta function match in the insert. The absolute t-parameter scaling was calibrated by measuring tune shift vs nominal t-parameter, as presented in Figure \ref{fig:dnTuneVsT}. The gap in the data between t=0.2 t=0.3 was an unfortunate result of the BPM system freezing and failing to update TBT data, and was simply excluded.

\begin{figure}
    \centering
    \includegraphics[width=0.8\linewidth]{./chapter_3_figures/dnDetuningCalFactor.pdf}
    \caption{Tune shfit vs t-parameter in both planes, before and after scaling}
    \label{fig:dnTuneVsT}
\end{figure}

The initial data indicated a discrepancy in the tune dependence of the t-parameter. The proportiona The ideal detuning expression was fit to the data with a t-scaling factor as the only free parameter Eq. (\ref{eq:fitTune}). 

\begin{equation}
    Q = Q_o \sqrt{1\pm 2at}
    \label{eq:fitTune}
\end{equation}

The fit resulted in a calibrated t-parameter scaling of $a \approx 0.935$. The source of this discrepancy is not clear, but may be a result of magnet crosstalk, as the relative calibration was measured with single elements and the magnet spacing is close.

\section{Momentum Reconstruction} \label{sec:momReconst}
For studying the evolution of the phase space, the four dimensional $(x,p_x,y,p_y)$ transverse coordinates of the beam need to be reconstructed. There are a few standard ways to accomplish this. The simplest approach is a pair of BPMs with a $\phi = \pi/2$ phase advance between them. We have more than two BPMs and would like to combine the signals. A common way to approach this is the N-BPM method, which reconstructs the momentum from measured beta functions. The most accurate beta function measurements are the so called beta from phase methods, where the relative phase advance between BPMs can adjust a model prior for the betas. In the case of IOTA with the NIO insert, the tune and therefore phase advance is highly nonlinear and not a reliably linear parameter. So instead we leverage the linear model of the lattice generated using LOCO. The momentum and position at a virtual BPM could then be reconstructing using a least squares approach. 

Before fitting the TBT data a few preprocessing steps were taken. BPM calibrations for scaling and roll as fit from the LOCO optimization were applied. A principle component analysis was applied to all of the BPMs. Based on the singular value, the first 8 components of the PCA were used to reduce the uncorrelated noise in the TBT signals.

Since the matching section of the DN system is most of the lattice footprint, all IOTA BPMs are located in this section. For the fitting, the matching section was essentially treated in a channel mode from the end of the nonlinear insert to the beginning of it, see Fig. \ref{fig:IOTAchannel}.

The practical IOTA lattice contains calibrated sextupoles for chromaticity manipulation, but expanding the method using second order transfer maps yielded poor quality fits of the motion, and strictly linear fits were used for analysis.

The error of the BPMs is evaluated to be the variation in the tails long after the signal has fully decohered. Note that the radiation damping time for the beam ~3 orders of magnitude longer than this sample time. So the individual particles are still and mostly full amplitude. For NIO lattice configurations starting 2000 turns from the end was selected to be well past the expected coherence limit. Figure \ref{fig:bpmErrCutoff} shows the full sample range of an example kick in IOTA with a line indicating the noise sample threshold.

\begin{figure}
    \centering
    \includegraphics[width=0.8\linewidth]{./chapter_3_figures/exampleKick_pointDN_t238.pdf}
    \caption{Full range IOTA example kick with error estimation cutoff}
    \label{fig:bpmErrCutoff}
\end{figure}

The error in the BPM measurments is then evaluated to be the variance in this remaining signal.

\begin{figure}
    \centering
    \includegraphics[width=1\linewidth]{./chapter_3_figures/tbtBpmNoise_iqrVstd.pdf}
    \caption{BPM noise over last 2000 turns, standard deviation and interquartile range markers included}
    \label{fig:bpmErrIQR}
\end{figure}

We now consider the effects of various preprocessing steps on the goodness of fit to the linear model. First we consider the linear BPM calibration factors from the LOCO procedure.

\begin{figure}
    \centering
    \includegraphics[width=0.8\linewidth]{./chapter_3_figures/redChiSquare_10_19_t238_rawVcal.pdf}
    \caption{Tune shfit vs t-parameter in both planes, before and after scaling}
    \label{fig:rawVcal}
\end{figure}

We now consider the effect of 

\begin{figure}
    \centering
    \includegraphics[width=0.8\linewidth]{./chapter_3_figures/redChiSquare_10_19_t238_rawVdropA1CandCalibrate.pdf}
    \caption{Tune shfit vs t-parameter in both planes, before and after scaling}
    \label{fig:rawVA1C}
\end{figure}

\begin{figure}
    \centering
    \includegraphics[width=0.8\linewidth]{./chapter_3_figures/redChiSquare_10_19_t238_fullPrepVdropB2L.pdf}
    \caption{Tune shfit vs t-parameter in both planes, before and after scaling}
    \label{fig:prepVB2L}
\end{figure}


\begin{figure}
    \centering
    \includegraphics[width=0.8\linewidth]{./chapter_3_figures/redChiSquare_10_19_t238_calAndDropVpcaClean.pdf}
    \caption{Tune shfit vs t-parameter in both planes, before and after scaling}
    \label{fig:calVpca}
\end{figure}

\begin{figure}
    \centering
    \includegraphics[width=1\linewidth]{./chapter_3_figures/10_19_t238_redChiSquare_dists.pdf}
    \caption{Tune shfit vs t-parameter in both planes, before and after scaling}
    \label{fig:redChiDists}
\end{figure}

The resulting fit can then be compared by threading through the same transfer matricies used to generate the reconstructed position and momentum.

\begin{figure}
    \centering
    \includegraphics[width=1\linewidth]{./chapter_3_figures/fitExampleBPM.pdf}
    \caption{Reconstructed position threaded through the rest of the machine}
    \label{fig:fitThread}
\end{figure}

\section{Kick Amplitude Calibration} \label{sec:kickAmpCal}

\section{Dynamic Aperture Evaluation} \label{sec:daEval}
The dynamic aperture (DA), or amplitude limit on stable motion, is a crucial metric of the real nonlinear system with full perturbations. The dynamic aperture is evaluated for simulated beams as the limit of asymptotically stable trajectories. For experimental measurements, the interpretation becomes more difficult. DA scans were performed by measuring the losses of kicked circulating beam. A consistent qualitative metric was applied, the loss limit was defined as a percent loss greater than 25\% or two subsequent kicks with percent losses greater than 10\%.

For a particular DA scan, a bunch was injected and scraped to a standard initial current in the BPM sensitivity range. The beam was iteratively kicked along fixed ratio "spokes" as current was logged. Once the current reached a minimum threshold, another bunch was injected and the scan of another spoke began.  Figure \ref{fig:spokes} illustrates the resulting geometry of the kick amplitudes.

\begin{figure}
	\centering
	\includegraphics[width=0.5\linewidth]{placeholder.pdf}
	\caption{Dynamic Aperture Scan configuration}
	\label{fig:spokes}
\end{figure}

Both the BPMs and DCCT were used to monitor beam current after a kick. The sum of the individual BPM button responses can be used as a fast indicator of losses. The BPM TBT output contains 20 turns of information before the kicker fires. For loss comparisons, the mean of the BPM signal for the first and last 20 turns was compared. The sample range of the BPMs was around 7000 turns which corresponds to a timescale of just under a millisecond. The DCCT signal was sampled before and after the kick on a timescale of a second. Loss timescales differed, some losses on kick were visible in the first few turns of the sum signal, and some losses were not visible in the sum signal but were visible in the DCCT. 

The noise in the DCCT signal was evaluated as the RMS noise with no circulating beam. The sum signal noise was evaluated as the RMS of the availible sample range.

To ensure that losses occurred in the time span between the DCCT samples, and were not dominated by the natural circulating beam lifetime, the difference between the second sample and the first sample of the next kick were calculated and verified to be at the level of noise in the DCCT signal. Figure \ref{fig:dcctKickLife}.


\begin{figure}
	\centering
	\includegraphics[width=0.5\linewidth]{placeholder.pdf}
	\caption{Change in DCCT current between DA scan kicks}
	\label{fig:dcctKickLife}
\end{figure}

Figure \ref{fig:daLimit} shows the DCCT and BPM sum signal for increasing kicks along a particular spoke. The red line indicates the determined DA limit for the particular spoke.

\begin{figure}
	\centering
	\includegraphics[width=0.5\linewidth]{placeholder.pdf}
	\caption{Percent Losses after individual kicks in a DA scan}
	\label{fig:daLimit}
\end{figure}

\section{Synchrotron Profiles} \label{sec:synchProfiles}

%\chapter{Experimental Studies} \label{chap:expResults}

\section{Working Point stability of Different NIO configurations} \label{sec:nioWorkPoint}
As discussed in section \ref{sec:nioDesign}, the NIO system requires chromatic compensation for stable operation. The circulating beam stability was investigated for different bare lattice configurations by observing current over time. Beam was injected into the bare NIO lattice and the strength of the NIO insert was slowly ramped. Constant beam losses stem from tousheck and intrabeam scattering effects, but significant drops in current or total beam losses are clear indicators of unstable configurations. Figure \ref{fig:bareRamp} shows the ramp for the naive bare lattice without sextupole compensation of any kind. We see complete losses near the horizontal third order resonance $3 Q_x = 1$. This is indicative of residual third order nonlinearities in the IOTA bare lattice. The source of these nonlinearities is not clear, but may stem from the dipole fringe fields, or the geometric nonlinearities from the tight bending radius in the dipoles. The chromaticity of the bare lattice also deviates from the model predicted value by a significant margin of over a unit, another indicator of spurious third order terms.

\begin{figure}
	\centering
	\includegraphics[width=0.8\linewidth]{./chapter_4_figures/2023-10-15aVstSlowRamp.pdf}
	\caption{Circulating current in IOTA while ramping t-parameter for no sextupole compensation}
	\label{fig:bareRamp}
\end{figure}


Chromaticity was then empirically compensated with a minimal set of two families of sextupoles. Figure \ref{fig:midRamp} shows a similar ramp after the compensation. The sextupole resonance is still present, but the stability is not strongly impacted and we can go all of the way to the integer resonance condition. We see a new loss location occur around $t=-0.46$, and some potential losses near $t=-0.3$. This motivates investigating the losses with a slowly varying t-parameter.

\begin{figure}
	\centering
	\includegraphics[width=0.8\linewidth]{./chapter_4_figures/2023-04-24aVstSlowRamp.pdf}
	\caption{Circulating current in IOTA while ramping t-parameter with chromatic compensation}
	\label{fig:midRamp}
\end{figure}


This slow ramp of the NIO insert is presented in Figure \ref{fig:slowRamp}. In this scan a second region of losses becomes more clear near the integer resonant condition. The nearest resonant line in tune space is $2Q_x + 2Q_y = 11$, in addition to a number of $6^{th}$ order lines. Whatever the cause of losses near this working point, it meant that this region had to be quickly stepped beyond when investigating dynamics at the integer resonance. Practically, the NIO insert was ramped to just above the limit of losses then ``snapped" beyond this point. There are no significant losses near $t=-0.30$ during the slower scan. This location corresponds to a third order coupling resonance, which does not drive losses but impacts turn-by-turn measurements.

\begin{figure}
	\centering
	\includegraphics[width=0.8\linewidth]{./chapter_4_figures/2023-05-02aVstSlowRamp.pdf}
	\caption{Circulating current in IOTA while slowly ramping t-parameter with chromatic compensation}
	\label{fig:slowRamp}
\end{figure}

Figure \ref{fig:fastRamp} shows quickly ramping does not incur significant losses. This is an improvement over the lattice tune in the previous runs. In the past, ad hoc sextupole knobs had to be adjusted as the insert was ramped to stably operate at various tunes. The fast ramp is the approach used in regular operation, an immediate ``snap" to a desired t-parameter proved too destabilizing. Additionally, a ramp provides for more iterative checking of the generally unreliable control system. There is no synchronization between individual power supplies, so small steps help to smooth out inconsistencias in the tim structure of the change. A particular family of power supplies also exhibited a proportional overshoot which favored small steps. This issue has been remedied with improvements to the supplies. This measurement was collected before the full DN calibration was finalized, so it does not quite arrive at the integer resonance equivalent setting like the scans above.

\begin{figure}
	\centering
	\includegraphics[width=0.8\linewidth]{./chapter_4_figures/2023-04-24aVstMidRes.pdf}
	\caption{Circulating current in IOTA while quickly ramping t-parameter with chromatic compensation}
	\label{fig:fastRamp}
\end{figure}


\section{NIO Studies Kicked Beam Collections} \label{sec:kickBeam}
Measurements of kicked beam data is convenient and information rich with electron operation. Synchrotron radiation damping means the beam occupies a relatively small portion of the phase space and the motion of the centroid closely approximates the motion of a single particle at the same amplitude. The radiation damping also means that the beam ``resets" after a kick as the beam damps back down to its equilibrium emitttance quickly. In the case of IOTA, the damping time is significantly longer than the BPM sample range, and on the order of the kicker reset timing. We do not introduce significant damping systematics in the TBT data, and can effectively kick as quickly as the hardware safely allows to maximize the utilization of the available lifetime between re-injections. For kicked beam measurements we have access to all 21 BPMs on a TBT basis and the current measurements. The synchrotron radiation cameras lack the acquisition systems and speed for kicked beam measurements, but the live monitoring provides useful operational information on kick success, amplitude, and damping. There were two primary configurations for kicked beam measurements in the experimental run. The first was a simple grid in the parameter space of the kicker setpoints. Early in the run, coarse calibration of the kickers was performed to evaluate their respective strengths. These ratios were used to roughly scale to a grid in beam configuration space. As described in section \ref{sec:kickAmpCal}, the absolute amplitude calibration of the kicks varies based on the exact lattice configuration, and drifts from collection to collection based on the power supplies. So, while the coarse calibration does not yield a uniform grid in configuration space, if the kick difference is small the density is sufficient to sample the full available phase space. Two grid ordering approaches were used. For fast measurements, the points were ordered according to the magnitude of their amplitude. This approach seeks to maximize the number of kicks per injection before beginning to lose beam on aperture, physical or dynamic. The second configuration of grid kicks used a raster approach, a single horizontal or vertical kicker setting was selected, and the perpendicular kicker iterated amplitudes until the beam was fully lost on the aperture. The kicks were rastered in both direction, first sweeping one direction then the other. This approach is necessarily slower, but yields the maximum range of samples in configuration space, and was only applied to a few lattice configurations which had been evaluated to have reasonable apertures and large detuning ranges. Figure \ref{fig:gridKick} shows an example of the kicker settings for a grid scan with current. The color scale is simply an incrementing counter of individual kicks. It shows the perpendicular rastering, we can see some cases where losses occur more quickly in one direction than the other, and the earlier kicks on the vertical raster lines peek out from behind. The two points not on the regular grid spacing are ``calibration" kicks used for collection purposes, but not analyzed.

\begin{figure}
	\centering
	\includegraphics[width=0.6\linewidth]{./chapter_4_figures/gridKickExampleMonochrome.pdf}
	\caption{Example control system kicker setpoints for a individual lattice configuration collection.}
	\label{fig:gridKick}
\end{figure}

The other primary collection for kicked beam measurement used was a spoke style scan intended for aperture measurements. For these scans a fixed kicker amplitude ratio was employed, with iteratively larger kicks applied along that ``spoke". Initially a binary search style method was evaluated for kick efficient measurement of the aperture. However, in practice, re-injections were time intensive and unreliable, so methods with expected losses at the end were preferable. For each spoke, the beam was re-injected, scraped to a consistent initial current, then kicked along the spoke starting at an expected lossless amplitude. The kick amplitude was then increased until total beam loss as measured by the DCCT. The method to measure the aperture with this approach is detailed in section \ref{sec:daEval}. In principle the lossless kicks from this approach could also be used for analysis of the TBT motion, but the much better statistics from grid-style scans means this was not often done.

There were two main sextupole configurations studied in the course of the TBT measurments. The first was the minimum chromaticity compensation setpoint mentioned above. To reduce the measured impact of the sextupoles on the dynamic aperture, an optimization of the sextupole configurations was undertaken. Using the remaining four families of sextupoles, four independent knobs which all preserved the first order chromatic compensation were generated and used as the variables for a bayesian optimization. The targets for this optimization were the apertures as measured by the logistic fits on the circulating curreint as described in section \ref{sec:daEval}. The resulting optimized sextupole configuration was nicknamed ``lilac" based on the color of trace in the optimization output, and has carried forward. The eperimental sextupole configurations following the IOTA naming conventions are given in table \ref{tab:sextupoleScaling}, in applied current and calculated integrated sextupole term. 

\begin{table}
    \centering
    \begin{tabular}{c|c|c|c|c}
       Sextupole Family & Chromatic I & Chromatic $K_3$ & Lilac I & Lilac $K_3$ \\
       \hline
       sa1  & 0 & 0 & -0.77 & -8.8127 \\
       sc1  & 0.805 & 9.2132 & 1.171 & 13.4021 \\
       sc2  & -2.11 & -19.6412 & -1.826 & -16.9975 \\
       sd1  & 0 & 0 & -0.263 & -3.0100 \\
       se1  & 0 & 0 & 0.254 & 2.9070 \\
       se2  & 0 & 0 & 0.81 & 9.2705 \\
    \end{tabular}
    \caption{Table of sextupole current settings and effective thin lens strength}
    \label{tab:sextupoleScaling}
\end{table}

\section{Conservation of Nonlinear Invariants} \label{sec:invConv}
A direct verification of proper implementation of the NIO system would be conservation of the analytically predicted invariants. The same approach was used in section \ref{sec:dnSims} for evaluation of insert configurations. To evaluate the invariant expressions, transverse momentum coordinates must be reconstructed as described in section \ref{sec:momReconst}, and the fitted coordinates needed to be normalized by the bare lattice Courant-Snyder functions at the virtual BPM location. Like the transfer matrices, these quantities were extracted from the design lattice fitted to by LOCO. To benchmark the method, the Courant-Snyder invariants were calculated for the same kicks. For reasonable comparison to the equivalent Courant-Snyder invariant, the first order effect on the lattice functions due to the NIO insert were simulated using quadrupoles in place of the full nonlinear elements. The calculations of the Courant-Snyder invariants used this normalization as opposed to the bare lattice functions. To avoid apparent invariant changes due to any linear coupling, the sum of the horizontal and vertical invariants (the normalized Hamiltonian) was used for comparison \cite[p.179]{leeAcceleratorPhysics2018}. Figure \ref{fig:invTbt} shows the resulting calculated invariant quantities TBT for two kicks. These parameters are normalized to the value at the first turn. Since the decoherence reduces the amplitude of the centroid signal, the invariant quantities also necessarily reduce. The TBT BPM position uncertainty also impacts the invariant uncertainty. The left plot shows a good result for measured conservation, where both nonlinear predicted invariants show relatively small deviation from their predicted values. The right plot shows a case where the second DN invariant is not conserved at all. We see large oscillations about a central value, this is a characteristic signature of an arbitrary value calculated from the dynamical variables. These responses were for the same lattice condition with different kick amplitudes.

\begin{figure}
	\centering
	\includegraphics[width=1\linewidth]{./chapter_4_figures/goodBadInvConsTbt_t238.pdf}
	\caption{TBT calculated invariant quantities, left plot shows general good conservation of all invariant quantities for initial calibrated amplitudes $x=1.34$ mm $y=1.44$ mm. Right plot shows poor conservation of DN second invariant for initial calibrated amplitudes $x=0.62$ mm $y=3.89$ mm.}
	\label{fig:invTbt}
\end{figure}


An approach to evaluating the variance of the predicted invariants is to look at the frequency spectrum of the values. For a conserved quantity, we expect the TBT noise to dominate and the frequency spectrum to be flat. For a non-conserved quantity, we expect peaks in the frequency spectrum corresponding to the oscillation. Figure \ref{fig:invLogLog} shows the FFT spectra for the same kicks as in figure \ref{fig:invTbt}. As the decoherence generates a strong zero frequency term in the spectrum, a log-log scale is used to emphasize the higher frequency spectral peaks. We can see the clear peak in the spectrum for the poorly conserved second invariant. This approach is useful as a graphical method, but does not yield an easily interpretable reduced quantity for a given kick.

\begin{figure}
	\centering
	\includegraphics[width=1\linewidth]{./chapter_4_figures/fourierInvariantCons_t238.pdf}
	\caption{Turn based spectral composition of calculated invariant quantities for example sets. Left plot for initial calibrated amplitudes $x=1.34$ mm $y=1.44$ mm, and right plot for initial calibrated amplitudes $x=0.62$ mm $y=3.89$ mm.}
	\label{fig:invLogLog}
\end{figure}

To compare different amplitudes and nonlinear insert configurations we calculate the standard deviation of these values over the first 28 turns. This was selected to maximize the number of points evaluated before the decoherence becomes prominent for the broadest range of sample kicks at the cost of significant uncertainty in the metric. The first 28 turns are highlighted in Figure \ref{fig:invTbt}. Figure \ref{fig:invt238} shows the variation of the Courant-Snyder Hamiltonian, DN Hamiltonian and DN second invariant over the first 28 turns versus the kicker setting. The lattice configuration is with the ``lilac" sextupole configuration and a nominal $t=-0.238$. The coupled calibration factors are used, which results in a skewed grid from the initial Cartesian input kicker settings.

\begin{figure}
	\centering
	\includegraphics[width=1\linewidth]{./chapter_4_figures/multitInvConsDeviation_t238.pdf}
	\caption{Color plot of standard deviation of analytically predicted invariants over first 28 turns, each plot has individual color scaling for better contrast but underlying metric is the same}
	\label{fig:invt238}
\end{figure}

The first DN invariant (the Hamiltonian) is similar in functional form to the overall Courant-Snyder invariant. As a result these two quantities display similar levels of conservation. A better metric is looking at the conservation of the second DN invariant. Based on this plotted invariant space we see reasonable conservation of both nonlinear invariants of motion for low vertical kicks and middling horizontal kicks. Based on this observation, a collection was taken for the same kicker settings repeatedly for the full beam lifetime. The initial calibrated amplitudes were selected to be $x=1.33$ mm and $y=1.29$ mm. Once again, the sextupoles are in the ``lilac" configuration and the nonlinear $t=-0.238$. We can observe the nonlinear invariants versus current for this setting, in Figure \ref{fig:pointCurrent}. There is a slight reduction in overall conservation as current goes down, this is an expected effect of the decreasing signal to noise ratio we see in the BPM TBT signals. Additionally, the values are all within the statistical uncertainty for the range from about 0.33 to 0.5 mA.

\begin{figure}
	\centering
	\includegraphics[width=1\linewidth]{./chapter_4_figures/invConsVcurr_point_t238.pdf}
	\caption{Invariant conservation for repeated kicks with same nominal amplitude as current naturally decays}
	\label{fig:pointCurrent}
\end{figure}

This set corresponds to the best conservation of the nonlinear invariants via this method. We can compare to the predicted deviation between the conservation of the Courant-Snyder invariant with adjusted lattice functions and the nonlinear invariants in simulation for the same initial amplitudes. Table \ref{tab:invConsv} shows the difference in the invariants in simulation and experiment. Two simulation conditions are considered, originally simulations were performed in the idealized linear lattice (``linmin" from \ref{sec:iotaSim}) indicating a clear difference in conservation. However, after identifying the best case conservation quantities, simulation fidelity had improved and comparative simulations with sextupole effects and expected nonlinearites (``nonlin" lattice) showed similar conservation quantites in simulation. Direct calculation of the analytical invariant quantities from fitted position data or simulation has insufficient resolution to demonstrate better relative conservation compared to the Courant-Snyder null hypothesis for a realistic lattice. We can still evaluate the topography of better and worse conservation for evaluting the interplay of NIO with perturbative nonlinearities.

\begin{table}
    \centering
    \begin{tabular}{lccc}
        \toprule
	\textbf{Invariant} & \textbf{Linmin} & \textbf{Nonlin} & \textbf{Experimental}\\
        \midrule
	CS Hamiltonian & \num{1.07e-2} & \num{5.04e-2} & \num{8.02e-2} \\
	DN Hamiltonian & \num{5.59e-3} & \num{4.95e-2} & \num{6.66e-2} \\
	DN Invariant & \num{6.83e-3} & \num{6.95e-2} & \num{7.54e-2} \\
        \bottomrule
    \end{tabular}
    \caption{Fractional Invariant conservation standard deviation for simulation and experiment with identical transverse initial conditions}
    \label{tab:invConsv}
\end{table}


\section{Amplitude Dependent Detuning} \label{sec:ampDetune}
A useful measurement of the nonlinearities of a system is to directly evaluate the change in tune with the amplitude. This figure of merit impacts the potential effectiveness of the Landau damping of a system. The amplitude dependent detuning was measured for different t-parameter settings ranging from $t=-0.05$ to $t=-0.41$. The resonant capture effects covered in sections \ref{sec:bunchSims} and \ref{sec:tune} had a strong impact on available detuning measurements beyond $t=-0.275$. Figure \ref{fig:t330Detuning} shows the detuning for a t-parameter setting of $t=-0.330$. Here the collection was a rastered grid-style scan of kicks, so many different amplitudes are sampled. Note the clustering of tunes on the third order coupling line, sextupole terms tend to drive this coupling and trap the beam on the resonance. This does not negatively impact stability, but renders us insensitive to the detuning effects from the NIO.

\begin{figure}
	\centering
	\includegraphics[width=1\linewidth]{./chapter_4_figures/zoomTuneSpreadErrorbar_t330_45turn.pdf}
	\caption{Amplitude dependent detuning for nominal $t=-0.330$, sampled for 45 turns}
	\label{fig:t330Detuning}
\end{figure}


Figure \ref{fig:t330phaseSpace} shows the resulting phase space for the beam on this coupling resonance. The contours are characteristic of the two-to-one ratio of the coupling tunes. While the amplitude dependent detuning from the nonlinear insert drives the beam centroid to the resonance, the resonance then dominates and we are insensitive to the NIO effects.

\begin{figure}
	\centering
	\includegraphics[width=0.8\linewidth]{./chapter_4_figures/coupledSextupolePhaseSpace.pdf}
	\caption{Reconstructed phase space for kicked beam captured on a second order coupling resonance}
	\label{fig:t330phaseSpace}
\end{figure}

The $Q_x = 1/3$ horizontal resonance line also impacted the available aperture for tune measurements. Figure \ref{fig:t140phaseSpace} shows the reconstructed phase space of a kick near this line for a NIO setting at $t=-0.140$. The characteristic triangular phase space of a third order resonance in the horizontal plane is visible. This is the same line which had to be crossed quickly when ramping the insert to higher values for collections.

\begin{figure}
	\centering
	\includegraphics[width=0.8\linewidth]{./chapter_4_figures/horizontalSextupolePhaseSpace.pdf}
	\caption{Reconstructed phase space for kicked beam with horizontal tune just above third order horizontal resonance}
	\label{fig:t140phaseSpace}
\end{figure}


Based on these constraints, to measure the maximum range of the detuning without resonant effects, a nominal working point of $t=-0.238$ was selected, which corresponds to $Q_x = 0.364, Q_y = 0.217$. Figure \ref{fig:t238detune45} shows the detuning for this point sampling 45 turns. This is quite fast and leads to some uncertainties in the tune measurements. Figure \ref{fig:t238detune45Error} shows a further zoomed version of the plot with errorbars indicating uncertainties of the tune measurements in each plane. The errors are vertically dominated as the decoherence in the vertical plane is much faster than in the horizontal plan. We still see resonant capture effects on fractions of the beam for this configuration. If we are dominated by TBT centroid signal before full decoherence we are sensitive to the DN NIO detuning. However, if the tune length is sampled for a long span, resonant capture of a fraction of the bunch begins to dominate the tune measurements. Figure \ref{fig:t238detune190} shows the same tune space measurements for a TBT sample range of 190 turns. We see resonant capture near sextupole and octupole coupling lines.

\begin{figure}
	\centering
	\includegraphics[width=1\linewidth]{./chapter_4_figures/zoomTuneSpreadGrid_t238_45turn.pdf}
	\caption{Amplitude dependent detuning for nominal $t=-0.238$, sampled for 45 turns}
	\label{fig:t238detune45}
\end{figure}


\begin{figure}
	\centering
	\includegraphics[width=1\linewidth]{./chapter_4_figures/zoomTuneSpreadErrorbar_t238_45turn.pdf}
	\caption{Amplitude dependent detuning for same data as figure \ref{fig:t238detune45} with a tighter zoom to emphasize the main range of tunes with uncertainties. Vertical uncertainties dominate due to very short decoherence times}
	\label{fig:t238detune45Error}
\end{figure}

\begin{figure}
	\centering
	\includegraphics[width=1\linewidth]{./chapter_4_figures/zoomTuneSpreadErrorbar_t238_190turn.pdf}
	\caption{Amplitude dependent detuning for nominal $t=-0.238$, sampled for 190 turns. Resonant capture dominates tune measurements for a large fraction of tune space}
	\label{fig:t238detune190}
\end{figure}


We can also compare this detuning with our directly calibrated amplitude measurements. As the amplitude calibration depends on the linear lattice parameters, the direct amplitude dependent measurements were only valid for the ``lilac" sextupole configuration. Figure \ref{fig:t238ampDetuneBase} shows the tune in both planes plotted against the equivalent initial linear action in both planes. Each plot contains every kick so the perpendicular detuning and amplitude is visible as the low amplitude tune spread for a given combination. Simulated tunes for the ideal IOTA lattice are plotted alongside. The tunes from simulation were calculated from single paricles initialized with identical transverse amplitudes as calibrated from the measurements. The overall direction of the detuning and the ranges are comparable, but a number of features are lacking. The horizontal and vertical detuning versus the horizontal amplitude are suppressed, the slope of the detuning versus the vertical kicks is different. We also see a slight mismatch in the origin of the detuning profiles, this indicates an imperfect bare lattice working point for the measurement configuration. This makes direct comparison of detuning a little harder, but supports the general stability of the NIO insert to perturbations in the matching lattice.

\begin{figure}
	\centering
	\includegraphics[width=1\linewidth]{./chapter_4_figures/impactxRun4TuneVampComparison.pdf}
	\caption{Measured centroid tune versus fitted initial linear action with simulated tunes in idealized IOTA lattice for identical initial actions}
	\label{fig:t238ampDetuneBase}
\end{figure}


We know that the bare lattice is more complicated than the ideal situation. We can introduce the sextupole nonlinearities consistent with full compensation of the chromaticities, the exact dipole mappings, and the nonlinear quadrupole fringe field effects to the lattice (``nonlin" lattice from section \ref{sec:iotaSim}). Figure \ref{fig:t238ampDetuneError} shows the same tune vs amplitude plots where the simulation contains these nonlinearities. The most immediately striking effect is the reduction in the range of the tune spread versus the horizontal amplitude, and the ``folding" effect which more closely matches the measured tune footprints.

\begin{figure}
	\centering
	\includegraphics[width=1\linewidth]{./chapter_4_figures/impactxRun4TuneVampComparisonErrors.pdf}
	\caption{Measured centroid tune versus fitted initial linear action with simulated tunes in IOTA lattice with ad hoc nonlinearities for identical initial actions}
	\label{fig:t238ampDetuneError}
\end{figure}

To illustrate the dominant impact of the NIO nonlinearites relative to the residual nonlinearites in the bare lattice we make one more comparison. Using the ``quadnio" simulation lattice, we replace the NIO insert lenses with equivalently scaled quadrupoles. This provides the proper first order change in lattice functions and working point. The rest of the lattice retains the expected nonlinearities in the bare configuration. In Figure \ref{fig:t238ampDetuneQuad} the same detuning vs amplitude plots show significant deviation favoring experimental measurement of the design NIO system. The detuning is much smaller without the nonlinearities in the NIO and the horizontal detuning is anti-correlated and depends on the perpendicular amplitude compared the DN detuning.

\begin{figure}
	\centering
	\includegraphics[width=1\linewidth]{./chapter_4_figures/impactxRun4TuneVampComparisonErrorsDNquad.pdf}
	\caption{Measured centroid tune versus fitted initial linear action with simulated tunes in IOTA lattice with ad hoc nonlinearities and first order component of DN NIO insert for identical initial actions}
	\label{fig:t238ampDetuneQuad}
\end{figure}

The effect is more striking in tune space, Figure \ref{fig:t238detuneQuad} shows the simulated tune spread for configuration S3. Here we can clearly see the anti-correlated horizontal detuning across the opposite side of the DN tune shift.

\begin{figure}
	\centering
	\includegraphics[width=1\linewidth]{./chapter_4_figures/impactXvmeasured_nonlin_quadDN_Detune_t238.pdf}
	\caption{Amplitude dependent detuning for simulated IOTA lattice with ad-hoc nonlinearities and first order component of DN NIO insert, in blue to red color. Measured detuning with NIO nonlinearities in green.}
	\label{fig:t238detuneQuad}
\end{figure}

We see significant impacts of the residual nonlinearities in the bare IOTA lattice. We can directly evaluate the amplitude dependent detuning for the bare ``lilac" as well. Figure \ref{fig:t0detune} shows this detuning. We see coherent detuning along the linear coupling resonance.

\begin{figure}
	\centering
	\includegraphics[width=1\linewidth]{./chapter_4_figures/zoomTuneSpreadErrorbar_t000_190turn.pdf}
	\caption{Amplitude dependent detuning for bare IOTA lattice. Evaluated for 190 turns, error bars are smaller than markers}
	\label{fig:t0detune}
\end{figure}

\section{Measured Aperture} \label{sec:DA}
For evaluations of the available aperture, we first need to understand the admittance, or accepted amplitudes of our lattice. This is a convolution of the apertures of the machine and the beta functions at each location. In IOTA the dominant minimizing aperture is that in the nonlinear insert. Due to the small size of the nonlinear magnet poles, the aperture is constrained, nominally by the NIO c-parameter scaled by the bare lattice beta functions. Figure \ref{fig:dnAperture} shows the CAD drawings of the IOTA NIO insert vacuum chamber, at the center and the entrance. The most strict minimizing requirements are horizontal so this lemon-shaped profile was adopted to simultaneously clear the poles and provide good vacuum conductance in the insert. 

\begin{figure}
	\centering
	\includegraphics[width=1\linewidth]{./chapter_4_figures/iotaApertureDrawingsSmall.png}
	\caption{CAD drawings of IOTA NIO insert vacuum chambers, with minimizing ellipses in apertures overlaid}
	\label{fig:dnAperture}
\end{figure}

Elliptical minimizing apertures at each location have been overlaid, while there may technically be extra vertical admittance beyond these ellipses, it is very narrow and simulation software typically only supports elliptical and rectangular apertures. The vacuum chamber also does not perfectly conform to the beta function scaling, so a single aperture physically defined is insufficient for the bare lattice. A combined minimizing aperture using the limits in each direction was calculated using the bare lattice beta functions. Table \ref{tab:dnAperture} gives the relative values for each. This is a bit coarse, but the uncertainties in our closed orbit location and beta function scaling mean that accuracy below 50\unit{\micro m} will be overprecision, so this combined minimizing aperture will be used for reference moving forward.

\begin{table}
    \centering
    \begin{tabular}{lccc}
    \toprule
    \textbf{Aperture} & \textbf{X Minor Axis} [mm] & \textbf{Y Major Axis} [mm] & \textbf{Axis Ratio} [1]\\
    \midrule
    Central & 3.84 & 5.15 & 0.75\\
    Edge  & 7.12 & 8.60 & 0.83\\
    Combined Central  & 3.84 & 5.05 & 0.76\\
    \bottomrule
    \end{tabular}
    \caption{IOTA NIO Insert Apertures}
    \label{tab:dnAperture}
\end{table}

There is an additional aperture impact on the admittance, the undulator used for other experiments has an atypical vacuum pipe which limits the vertical admittance for low t-parameter values. The first order effect of the DN insert can be used to adjust the beta functions, so we can see that this restriction goes away and the insert vacuum becomes the minimizing aperture for t-parameters $t<-0.2$. Before then, though we end up with a superposition of the ellipse and a flat vertical limiting aperture.

To evaluate the different sextupole configurations, full aperture scans of the bare lattice were taken for different sextupole configurations after the optimization. The losses along fixed amplitude ratio ``spoke" kicks were taken as outlined in section \ref{sec:daEval}. Figure \ref{fig:bareDA} shows the resulting measured aperture limits. The apertures are symmetric, so the limits are mirrored twice to give a sense of the aperture in configuration space. For comparison, the admittance scaled to the center of the nonlinear insert is plotted in the thick black lines. Both the insert vacuum and the undulator contribute here, so the actual admittance restriction becomes the minimum of these overlapping contours, a vertically truncated ellipse. The uncertainties presented are the rms beam sized according to the expected emittance for the current at loss combined with a 50\unit{\micro m} uncertainty from the kicker amplitude. The flat vertical aperture also includes a contributed uncertainty (represented in the grey shading) from the closed orbit in the undulator. While the closed orbit is carefully controlled in the nonlinear insert, it is less well controlled in the rest of the lattice, which contributes to about a $\pm 0.5$mm uncertainty in the closed orbit at that aperture. We see significant impacts on the aperture from the sextupole configurations. The vertical restriction is consistent between the bare and chromatic lattices and near the undulator aperture. Potentially, this means that the aperture metric systematically under evaluates physical aperture restrictions. This is not so much of a limitation for evaluating relative dynamical losses, so it was not pursued further. But, it does serve to emphasize that the greatest value from these loss scans is in their relative features, and not in the exact loss limits. Adding the minimum chromatic compensation sextupole complement does not adjust the aperture much, making it slightly larger. The optimization result with the ``lilac" configuration shows significant gains horizontally at the cost of some vertical aperture.

\begin{figure}
	\centering
	\includegraphics[width=0.6\linewidth]{./chapter_4_figures/sextupoleLossPhysical.pdf}
	\caption{Aperture limits of different sextupole configurations for the IOTA bare lattice}
	\label{fig:bareDA}
\end{figure}

Figure \ref{fig:lowTlilacDA} shows the evolution of the aperture for a few low t-parameters in the ``lilac" configuration. We can see here that the vertical admittance changes with the t-parameter, so they can not be strictly plotted on top of each other. We see significant horizontal restriction at $t=-0.094$. There is also significant vertical restriction at $t=-0.14$. Both of these collections are near the horizontal third order resonance, and the dynamic aperture restrictions are consistent with detuning onto the unstable sidebands of this resonance.

\begin{figure}
	\centering
	\includegraphics[width=1\linewidth]{./chapter_4_figures/lilacLossVertAperture.pdf}
	\caption{Aperture limits for the ``lilac" sextupole configuration with different t-parameters, in the range where the vertical aperture limitation in the undulator still contributes to the admittance}
	\label{fig:lowTlilacDA}
\end{figure}

Figure \ref{fig:lilacDA} shows the evolution of the aperture for a selection of t-parameters beyond where the admittance is affected by the undulator vacuum, so the contours are all restricted by the same minimizing aperture in the nonlinear insert. We see reasonable aperture conservation in the vertical plane for the first few t-parameters, but the horizontal aperture shrinks to a plateau for middling t-parameters around our nominal good configuration. For larger t-parameters beyond $t=-0.33$ we begin to see a significant reduction in the aperture, and approaching a t-parameter of $t=-0.45$ the kicked beam losses method becomes unreliable due to the working point losses in this region.

\begin{figure}
	\centering
	\includegraphics[width=0.6\linewidth]{./chapter_4_figures/lilacLossPhysical.pdf}
	\caption{Aperture limits of different t-parameters for the IOTA ``lilac" lattice}
	\label{fig:lilacDA}
\end{figure}

Figure \ref{fig:chromDA} shows the evolution of the aperture for different t-parameters in the IOTA lattice with only chromatic compensation sextupoles. The $t=0$ measurement here is not the same as in figure \ref{fig:bareDA} as collections from the same day are preferred for the most direct comparison. There are a number of aperture measurements for t-parameter settings below $t=-0.2$, but only the $t=0$ setting is constrained by the undulator admittance, so only that limiting aperture is plotted instead of the separate plots in figure \ref{fig:lowTlilacDA}. Its relative value is not applicable for the rest of the aperture contours, e.g. the $t=-0.187$ is not near the undulator aperture at its vertical admittance settings. Here we see that the evolution is largely the same as with the ``lilac" sextupole configuration. The horizontal aperture shrinks before the vertical, and both axes shrink significantly for values of the t-parameter beyond $t=-0.3$. Unfortunately, a direct comparison cannot be made since not all of the same t-parameters were sampled between these lattice configurations. Ultimately, based on the reduction of aperture in both of the widely sampled sextupole configurations near sextupole resonances, it seems that the simple optimization of dynamic aperture for the bare lattice is an insufficient condition for maximizing the available NIO aperture.

\begin{figure}
	\centering
	\includegraphics[width=0.6\linewidth]{./chapter_4_figures/chromaticLossPhysical.pdf}
	\caption{Aperture limits of different t-parameters for the IOTA bare lattice with a minimum set of chromatic compensation sextupoles}
	\label{fig:chromDA}
\end{figure}

Of particular interest for the stability of the DN NIO implementation is its robustness to linear perturbations in the matching lattice. To evaluate this stability a number of orthogonal perturbations to the linear matching lattice were constructed, with a particular focus on the lattice functions in the nonlinear insert drift. To evaluate the relative effect, a consistent t-parameter of $t=-0.223$ and the ``lilac" sextupole configuration were applied for all collections. The t-parameter was intended to be the nominal $t=-0.238$ as used for the amplitude dependent detuning and invariant conservation measurements. Unfortunately, by mistake, the calibration factor found in \ref{sec:nioCal} was not applied. Luckily a consistent value of $t=-0.233$ was applied for almost all of the aperture measurments, so we can still make relative measurements. As a baseline, we should evaluate the stability of the aperture measurment for collections at different times. Figure \ref{fig:t223DA}, shows the nominal lattice aperture at $t=-0.223$ on different collection days. The 10/07 collection shows a little inconsistency vertically, but the horizontal apertures are consistent, and we will compare the measurements to the nearest base lattice which excludes the 10/07 collection moving forward.

\begin{figure}
	\centering
	\includegraphics[width=0.6\linewidth]{./chapter_4_figures/t223LossPhysical.pdf}
	\caption{Aperture limits for $t=-0.223$ with ``lilac" sextupole settings on multiple collection days}
	\label{fig:t223DA}
\end{figure}

The first comparison is adjusting the phase advance in the matching lattice while leaving all the lattice functions exactly the same in the insert. This exact knob was used on a daily basis for matching the tunes of the bare lattice, usually to a $\Delta Q$ lower than \num{10e-3}. The tune in both planes could be adjusted independentlyThe perturbations are then significantly worse than the level of lattice control we expect. We see general reduction of the aperture for mismatch of the external lattice, with slightly more losses for the increased tune.

\begin{figure}
	\centering
	\includegraphics[width=0.6\linewidth]{./chapter_4_figures/tunePerturbLossPhysical.pdf}
	\caption{Aperture limits for perturbations of the bare lattice phase advance in the matching section outside of the insert with $t=-0.223$}
	\label{fig:tunePerturbDA}
\end{figure}

The next comparison is adjusting the phase advance across the nonlinear insert, in this case the integer matching condition is conserved, so the overall tune still changes. The system is quite robust to these adjustments, with a sizeable reduction only for significantly reducing the phase advance. Recall that the overall phase advance through the insert is $\Phi_x = \Phi_y = 0.3$ in tune units, so these perturbations are a significant fraction of the overall phase advance. This can be partially understood as a poor implementation of the integration, for a symmetric mismatched phase, it simply looks like we did a worse job of matching the nominal scaling with the bare lattice beta function.

\begin{figure}
	\centering
	\includegraphics[width=1\linewidth]{./chapter_4_figures/phiPerturbLossPhysical.pdf}
	\caption{Aperture limits for perturbations of the bare lattice phase advance across the nonlinear insert with $t=-0.223$}
	\label{fig:phiPerturbDA}
\end{figure}

The next two perturbations we consider are the location of the minimum of the beta function $\beta^*$ longitudinally in the insert. These knobs are quite small, nominally 9 mm in a drift of 1.8 m, but we already see significant reduction in the aperture with these samples. This was one of the main knobs used for increasing the beam lifetime at the integer resonance discussed in section \ref{sec:intCross}. The second knob considered here is the amplitude of the dispersion function through the NIO insert. Once again these knobs are quite small, but we see some reduction of the aperture for the most aggressive knob. This is the only comparative perturbation collection with non $t=-0.223$ conditions, the two last points in the dispersion comparison were taken at the properly calibrated $t=-0.238$. However, we see no significant reduction in aperture for the $+\num{2e-2}$ m $D_x$ collection, so the comparison is likely reasonable.

\begin{figure}
	\centering
	\includegraphics[width=1\linewidth]{./chapter_4_figures/latPerturbLossPhysical.pdf}
	\caption{Aperture limits for perturbations of the $\beta^*$ location and the dispersion across the nonlinear insert}
	\label{fig:latPerturbDA}
\end{figure}

Additionally, an effect related to the configuration space of the NIO system may shrink the admittance. For larger t-parameters, the configuration space begins to take on an ``hourglass" shape, so for a given horizontal axis position, parts of the beam off the median symmetry line may have larger horizontal offsets and be lost on the minimizing aperture. Figure \ref{fig:dnAdmitEvolve} shows the case for a simple simulation (``toy" lattice in section \ref{sec:iotaSim}) of only the nonlinear insert and a matching matrix. We see losses on the central aperture changing near the horizontal axis for large displacements. This is a strictly nonlinear effect, for a linear system the admittance for this single aperture case would simply be the aperture scaled by the beta function. This effect was seen to begin to impact the aperture only at t-parameters beyond $t=0.-3$, and the measured apertures were already significantly reduced in these conditions, so it is unlikely to impact the presented measurements. For future experiments, a full aperture model of IOTA should be constructed and evaluated with DN impacts in tracking simulations, but this is unfortunately beyond the scope of this dissertation work.

\begin{figure}
	\centering
	\includegraphics[width=1\linewidth]{./chapter_4_figures/dnAdmittanceEvolution.pdf}
	\caption{Evolution of admittance for a simple thin lens DN system}
	\label{fig:dnAdmitEvolve}
\end{figure}


\section{Nonlinear Stability at Integer Resonance} \label{sec:intCross}
The synchrotron camera images were used for evaluation of beam stability at the integer resonance condition. To make this observation, the t-parameter was incremented while logging the time dependence of the synchrotron radiation profiles. Based on the calibration of the t-strength from the linear working point in section \ref{sec:nioCal}, settings crossing the vertical integer resonance were selected. The linear system is fundamentally unstable at this tune, but the NIO system retains stability. Measurements supporting stability in these conditions are strong evidence for the NIO system. To evaluate stability, the lifetime of the beam was evaluated at each t-parameter setting. Beam current was low due to a restricted dynamic aperture and normal beam lifetime, so the DCCT measurements were insensitive and intensity was evaluated from the synchtrotron images. Section \ref{sec:synchProfiles} describes the methods to extract the lifetime from the synchtrotron images.

The topography of the DN system is useful here in tuning the measurements and determining the working point. At strengths of the nonlinear insert beyond the integer resonance, the origin of the system becomes an unstable fixed point. Two new stable fixed points form above and below the origin. The result in the machine is the slow splitting of the beam into two stable ”beamlets” about these new fixed points. By evaluating when the beam becomes fully split, an upper limit on the crossing of the integer resonance location can be set. Figure \ref{fig:dnPotSplit} shows the analytical potential contours for t-parameters crossing the vertical integer resonance. We see the new minima split and move vertically away from the origin.


\begin{figure}
	\centering
	\includegraphics[width=1\linewidth]{./chapter_4_figures/dnPotVsToverInteger.pdf}
	\caption{Analytical potential contours crossing integer resonance}
	\label{fig:dnPotSplit}
\end{figure}


Figure \ref{fig:allCams} shows the raw images in all of the cameras for a t-parameter just beyond the integer resonance condition. Here we see the characteristic distribution about the two new fixed points. Based on the calibration of the step size between t-parameter settings and the location of the splitting of the beam, we can attach a nominal t-parameter of $t=-0.502\pm0.002$ to these profiles. The asymmetry in the beam distribution is another characteristic feature of operation at and beyond the integer resonance with the DN NIO system. Small deviations in the closed orbit means the beam distribution tends to oscillate between the two main potential wells over relatively slow timescales, yielding a visible ``flickering" of the distribution during operation. By evaluating many profiles over time for the same t-parameter, a more even distribution can be obtained, and is the approach used moving forward.

\begin{figure}
	\centering
	\includegraphics[width=1\linewidth]{./chapter_4_figures/allCamst525.png}
	\caption{Example sychrotron radiation profiles for all cameras, t-parameter $t=-0.502$}
	\label{fig:allCams}
\end{figure}

In addition to evaluating the integer resonance, the synchrotron profiles could be used for fine tuning the closed orbit through the insert. Varying the closed orbit knobs to arrive at an equal distribution of beam in the top and bottom beamlets is a clear indicator of proper alignment about the center of the insert. Additionally, the characteristic lattice functions in the nonlinear insert could be adjusted. Figure \ref{fig:synchCenter} shows a comparison of the synchrotron profile of the beam for two different manual optimizations. The left plot is the nominal IOTA lattice with best manual optimization for beam centering in the nonlinear insert. Here many images at the same t-parameter are combined to smooth out this slow scale flickering mentioned above. The plot on the right shows the same set t-parameter optimized for best lifetime, here the beam is less centered in the DN potential. Less obvious from this plot is the fact that the t-parameter scaling suffers some drift, the location of the integer resonance based on the topology changes with respect to the setpoint t-parameter for different collections. The strong dependence of the tune on the t-parameter and the resolution and repeatability of our power supplies means the topology is a more accurate indicator of the resonance crossing than the lower t-parameter calibrations in this regime. Additionally, the strong suppression of the dynamic aperture in these regions, means kicked beam tune measurements are unavailable.

\begin{figure}
	\centering
	\includegraphics[width=0.8\linewidth]{./chapter_4_figures/synchLightNomOptM3R_t525.pdf}
	\caption{Sychrotron radiation profiles for the nominal lattice on the left, and lattice optimized for lifetime on the right, in camera M3R, t-parameter $t=-0.502$}
	\label{fig:synchCenter}
\end{figure}

Figure \ref{fig:synchLifeNom} shows the fitted lifetimes for the beam as the integer resonance is crossed for the well centered lattice configuration. The red circled points correspond to the synchtrotron profiles plotted in Figure \ref{fig:intCrossSynchNom}, which were used to set the upper limit on the integer resonance crossing indicated in the green bar, around the nominal t-parameter of $t=-0.5$. Also included for reference are some t-parameters beyond the splitting point, where two separately stable orbiting beamlets propagate in the machine. The camera exposure had to be changed during the course of the measurements, as continuing beam loss reduced the signal. This means that only the relative intensity for a given exposure setting could be evaluated, and the absolute circulating current values were not available.

\begin{figure}
	\centering
	\includegraphics[width=1\linewidth]{./chapter_4_figures/integer_crossing_lifetimes_nominal.pdf}
	\caption{Fitted lifetimes from synchrtron radiation intensity for beam crossing integer resonance, in nominal IOTA lattice with manual centering in NIO potential.}
	\label{fig:synchLifeNom}
\end{figure}

\begin{figure}
	\centering
	\includegraphics[width=1\linewidth]{./chapter_4_figures/m1l_split_tight_aspect_tlabel_nominal.pdf}
	\caption{Synchrotron radiation profiles for crossing the integer resonance, the original stable fixed point splits to two new fixed points which move away from the origin with the t-strength.}
	\label{fig:intCrossSynchNom}
\end{figure}

Figure \ref{fig:synchLifeOpt} shows the fitted lifetimes for the beams at the integer resonance after manual optimization for best lifetime. We see some significant improvement for lifetimes up to and at the integer resonance limit. Once again, the red circled points correspond to beam profiles in Figure \ref{fig:intCrossSynchOpt}. Here a slightly broader range of nominal t-parameters is selected to show the full transition from the almost perfectly elliptical profile through to the widely separated beamlets. As the orbit centering is imperfect in this lattice we can see an asymmetric current distribution in the beamlets, as the lower beamlet in the last plot contains a larger fraction of the total circulating beam.

\begin{figure}
	\centering
	\includegraphics[width=1\linewidth]{./chapter_4_figures/integer_crossing_lifetimes_optimized.pdf}
	\caption{Fitted lifetimes from synchrtron radiation intensity for beam crossing integer resonance, in lattice optimized for best lifetimes near integer resonance.}
	\label{fig:synchLifeOpt}
\end{figure}

\begin{figure}
	\centering
	\includegraphics[width=1\linewidth]{./chapter_4_figures/m1l_split_tight_aspect_tlabel_optimized.pdf}
	\caption{Synchrotron radiation profiles for crossing the integer resonance, the original stable fixed point splits to two new fixed points which move away from the origin with the t-strength.}
	\label{fig:intCrossSynchOpt}
\end{figure}

The lifetime measurements on the order of minutes corresponds to many millions of turns, and indicates asymptotic stability of particles at the integer resonance where the nonlinear focusing terms dominate.

\chapter{Summary and Conclusion} \label{chap:end}

The following is a brief summary of my major developments in experimental beam dynamics studies in IOTA and contributions to a deeper understanding of nonlinear integrable optics for particle accelerators. The experimental measurements in IOTA provided substantial new data on the NIO system, and motivated an extensive set of supporting simulations for further understanding of the dynamics.

Kicked beam measurements of the low emittance electron beam allowed turn-by-turn reconstruction of the dynamics in the accelerator. Fast decoherence due to strong amplitude dependent detuning from the NIO system complicated measurements of relevant quantities and required extensive verification of analysis algorithms. Supporting tracking simulations further verified that measurements of the centroid of the realistic IOTA beam closely matched the action of a single particle with identical amplitudes, even with strong nonlinearites. Virtual BPM reconstruction was implemented for analyzing the phase space and calibrating the amplitude of individual kicks. Initially, direct verification of conservation of the analytically predicted invariant quantities was attempted. The current IOTA configuration demonstrated clear variance in this quantity, especially with regards to the second invariant of the NIO system. Further investigating the regions of best conservation yielded results which indicated insufficient sensitivity to definitively demonstrate this conservation. Supporting tracking simulations with nonlineaities of the form expected in the bare IOTA lattice were similarly insensitive to this exact conservation evaluation consistent with the measured quantities. Amplitude dependent detuning measurements showed a significant tune footprint and impacts of perturbative nonlinearities in the form resonant capture of fractions of the bunch were well measured. Initial deviations from the predicted tune footprint were remedied with simulations including the expected residual nonlinearites.

The transverse aperture of the system was measured using kicked beam losses. The bare lattice aperture showed significant effects from the sextupole configuration in the lattice, and an effort to minimize this was undertaken. Evaluation of the aperture for different relative strengths of the integrable nonlinearites showed a slight reduction in aperture for a broad range of the strength, comparable to the aperture effects of the base sextupole configurations. For relative nonlinear strengths beyond two-thirds of the tune shift to the integer resonance, the aperture reduced significantly. Measurements of the aperture of the NIO system with various linear matching lattice perturbations demonstrated good robustness to a variety of knobs, indicating that the system is generally stable to typical lattice perturbations.

By increasing the relative strength of the integrable nonlinearities in the system, the expected shift in the working point to the vertical integer resonance was demonstrated. In this configuration, the unique nonlinear focusing dominates, and the circulating beam is not entirely lost. Nonlinear perturbations and imperfections in the implementation of the NIO potential means that the dynamic aperture shrinks substantially in this condition. However, the circulating beam within this aperture demonstrates measurable macroscopic lifetimes on the order of minutes. This corresponds to millions of turns and is indicative of asymptotic stability of particles within this aperture. This stability is only possible due to the nonlinear focusing from the NIO insert. Additionally, by further increasing the strength of the nonlinearites, the central fixed point on the closed orbit splits into two separate orbits which move vertically away from the original point. This was measured as two separate, stable ``beamlets" which moved apart with increasing nonlinear insert strength. This effect is a unique characteristic of the NIO system studied and is direct support of reasonable matching to the NIO condition.

Overall, strong evidence of the unique dynamics of the NIO system was observed, though detuning, the topology of the phase space, and unique stability and behaviour beyond the integer resonance. The system proved to be reasonably robust and stable in operation, even with significant linear and nonlinear perturbations. The strong detuning and reasonable aperture of the NIO system are promising steps towards implementation in an operational user-focused accelerator. Importantly, the dominant shortcomings of the current configuration in the form of strong sextupole perturbations were identified as a significant limiting factor on the NIO system, which will be leveraged in further experiments at IOTA.


%\begin{appendices}
%\chapter{Relationship between canonical momentum and energy deviation} \label{apx:delToPt}

We would like to briefly consider the exact relationships between the canonical longitudinal coordinates $p_t$ and $\delta$

%\chapter{Second Generation Danilov-Nagaitsev Insert Magnet} \label{apx:dnv2}

The second iteration of the DN integrable insert makes a few different design decisions from the first. The largest change is a reduction from 18 equally spaced lenses to 11 in an equi-phase spaced configuration. As discussed in \ref{sec:dnSims} we see similar quality of conservation of the analytically predicted invariants in this configuration with fewer elements. In this configuration, the central elements are more closely packed, and the relative integrated strength of the elements becomes the same. Figure \ref{fig:dnv2int} shows the integration of the nominal potential with the equi-phase elements. The dense central packing requires differently shaped individual lenses, with shorter magnets in the center.

\begin{figure}
	\centering
	\includegraphics[width=0.6\linewidth]{./appendix_figures/dnIntPiecewiseEquiPhi.pdf}
	\caption{Integration steps for equi-phase insert}
	\label{fig:dnv2int}
\end{figure}

The other major change in the insert design is a signifcantly icreased physical aperture. There are two ways to accomplish this, the first is increasing the geometric DN c-parameter. This naturally increases the radius of the good field region, but also effectively reduces the nonlinearities the beam is sensitive to. In the case of the new insert, the c-parameter is increased to $c=0.014 \sqrt{\mathrm{m}}$. The other adjustment is changing the contour in the magnetic potential used for the pole face. Contours further from the origin can be selected, which naturally increases the minimum aperture. For the second iteration of the magnet, two contours were used. For magnets near the center, a contour at a magnetic scalar potential of 0.79, normalized by the rigidity $B\rho$. This is to maximize the avialable aperture. For magnets further out on the edge, a scalar potential of 0.653 was used as the aperture restrictions are lesser here in the equi-phase configuration. Figure \ref{fig:dnv2poles} shows the comparison of the potential contours selected for the new insert as compared to the contour for the original insert in normalized coordinates. The circle where the potential yields the dynamics we are interested in is also plotted for comparison.

\begin{figure}
	\centering
	\includegraphics[width=0.8\linewidth]{./appendix_figures/dnMagnetContoursV1andV2.pdf}
	\caption{Potential contours for inner and outer magnets on new insert compared with potential contour for first insert.}
	\label{fig:dnv2poles}
\end{figure}


Finally, the design of the vacuum chamber was changed from a complicated smootly tapering structure to stepped cylinders of constant radius. As of writing this dissertation, the vacuum chamber is still being manufactured, so the exact final dimensions cannot be verified.

%\chapter{Octupole Magnetic Fiducialization} \label{apx:oct}

This covers the magnetic field measurments for fiducialization and centereing of the octupole insert.

%\chapter{Sextupole Errors} \label{apx:sextErr}

When optimizing the different sextupole configurations, it is important that the sextupoles are well aligned to minimize the impacts of the changing excitations to the linear lattice functions. This minimizes systematics in the evaluation due to chanages in relative beta function, path length, etc. For a static sextupole configuration, a round of LOCO can be run again and any remaining linear lattice impacts accounted for. Before running the optimization, each sextupole was excited to evaluate the closed orbit shift. A few sextupoles were found with outsized impacts on the closed orbit location. To increase the sensitivity of the relative measurements, a number of closed orbit bumps were constructed in the relevant sextupole locations. The relative closed orbit impacts vs excitation at the full range of these bumps was measured. Assuming local linearity of the fields, the effective dipole field was calculated for these bumps. By mapping these evaluated fields to the bump locations in the magnet, an approximate field mapping can be made of the element. 

Figre \ref{fig:unalignedSext3D} shows the evaluated ``normal" dipole field against the nominal orbit offset in an unaligned sextupole. Note the general saddle shape of the plot with positive quadradic dependence in $x$ and negative quadratic dependence in $y$. But, our magnetic center at the fixed point of the saddle is positioned well away from the center. While the closed orbit response to the excitations is noninar, the linear approximation of the fields would still preserve the same minimum of the quadratic dependence on bump positions. The center of quadratic fits of the fields vs the bump location indicates the offset to be mechanically shifted. Notably, a constant dipole term had to added to improve the goodness of fit to an acceptible level. This indicates a fundamental dipole error term in the relevant magnet.

\begin{figure}
	\centering
	\includegraphics[width=0.7\linewidth]{./appendix_figures/sd1lB13D.pdf}
	\caption{Measured field from closed orbit responses in an unaligned sextupole for different transverse closed orbit bumps. 2D projected points are the same measured fields with errors.}
	\label{fig:unalignedSext3D}
\end{figure}

The worst case alignment magnets were physically moved and the measurements repeated to evaluate the method. Figure \ref{fig:sextField3D} shows the same measumrents in an aligned sextupole. The saddle shape remains, but the fixed point is centered at the origin, demonstrating reasonable effectiveness of the approach. Only ``normal" dipole terms are presented as significant vertical orbit deviation was not an issue. This method produced reasonable results for the evaluated magnets, and clearly identified the elements with residual dipole terms, but was quite time consuming and sensitive to the calibration of the bumps in the magnets. 

\begin{figure}
	\centering
	\includegraphics[width=0.7\linewidth]{./appendix_figures/sd1lB13Dcentered.pdf}
	\caption{Measured field from closed orbit responses in an well aligned sextupole for different transverse closed orbit bumps. 2D projected points are the same measured fields with errors.}
	\label{fig:sextField3D}
\end{figure}

After the experiment, an example dipole which indicated residual dipole field was removed to be evaluated on the magnetic measurment stand described in \ref{apx:oct}. Excited measurments of the magnet indicated substantial dipole componentes after centering the multipole decomposition on the sextupole term, plotted in Figure \ref{fig:sextMult2A}.

\begin{figure}
	\centering
	\includegraphics[width=0.6\linewidth]{./appendix_figures/1AMultipoles.pdf}
	\caption{Multipole decomposition for sextupole with dipole error term.}
	\label{fig:sextMult2A}
\end{figure}

The resulting field map in Figure \ref{fig:sextMap} makes the residual dipole term clear as the field lines crossing the x-plane.

\begin{figure}
	\centering
	\includegraphics[width=0.6\linewidth]{./appendix_figures/1Amap.pdf}
	\caption{Measured field map for sextupole with dipole error. Lines indicate $B$ field direction, colors magnitude of field. Red is magnetic north pole, blue is south.}
	\label{fig:sextMap}
\end{figure}

To evaluate the impact of hysteresis, the magnet was deguassed by oscillating between positive and negative excitations with a steadily decreasing amplitude of excitation. After the degaussing procedure, the magnetic multipole composition was measured  without excitation and found to be on the order of the uncertianty of the hall probe. Exciting the magnet again reproduced the relevant dipole fields. This indicates a fundamental error in the magnet construction. Comparative simulations of the fields with deliberate errors were made in Mermaid (magnetic simulation software developed at Budker institute), consisting of single pole offests, magnet half misalignment. and missing turns. These simulations and mechanical measurements exhonerated simple mechanical misalignments as sufficient to cause the measured dipole term. The only remaining potential contributions are miswound coils or poles with significantly different relative permiability. The sextupoles in question were prototypes, and the neccesary vetting of steel permeability may not have been done. Regardless, the straightforward remedy for both errors are simply winding auxilary coils on the offending poles and powering them alongside to even the field.

%\end{appendices}

%\printbibliography

\end{document}
