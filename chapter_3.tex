\chapter{Experimental Analysis} \label{chap:analysis}


\section{Tune Measurements} \label{sec:tune}
Tune measurments provide long term coherent information on the dynamics and the strong dependence on amplitude is the chief experimental indicator of nonlinearities in the focusing. While decoherence limits the available turns for measurments, the variations in amplitude do not alter the fundamental frequency, which gives us a longer coherent signal. The tune was previously defined only for the linear system in the context of phase advance. The definition of the tune is more nebulous then for a strongly nonlinear system, as the motion is neccesarially anharmonic. For nonlinear dynamical systems, the poincare rotation number is a related quantity \cite{nagaitsevBetatron}, but difficult to relate to measurable quantities. For the experimental measurments, we interpret the dominant frequency of the transverse oscillation spectrum as the tune. 

Different tune measurment algorigthim were considered for the particular limitations of the available data. The simplest approach is simply taking the Fourier transform of turn by turn (TBT) bpm data and picking the peak frequency. As we have discrete sampling of the motion, the fast fourier transform (FFT) is the default approach. There are a number of methods to improve the resolution of the frequency measurements. In principle a windowing function can be applied to improve resolution of the peak frequency at the cost of supressing sidebands. In the case of the short coherent measurments, the reduction in available signal was more detrimental than the potential advantages. To bypass some of the limitations on the disctrete sample resolution of the FFT, Jacobsen interpolation \cite{jacobesnLocla} applies a quadratic interpolation to the FFT peak and its nearest neighbors for finer peak resolution. The implmentation used is based on that from the PyLHC \cite{cernomc} analysis library. The other approach considered is the Numerical Analysis of Fundamental Frequencies (NAFF) proposed by Lasker for the evaluation of long term stability of planetary systems \cite{laskar}. NAFF seeks to iteratively optimize the magnitude of a single fourier transform, or equation \ref{eq:naff}. In principle this is used to extract successive harmonics of the motion, but we are interested only in the dominant frequancy, so typically only one term is considered. In addition to the offline methods, live tune measurments on circulating beam used a least squares approach, where an assumed functional form is fit to the tune, phase, coupling, and decoherence times.

\begin{equation}
	G = \int dq
	\label{eq:naff}
\end{equation}

To evaluate the methods, a comparison was made with both a synthetic signal and the simulated bunch centroid signal from an Impact-X simulation. Figure \ref{fig:syntune} shows the evaluation signals, the amplitude does not impact the tune evaluation. The synthetic signal consisted of a convolution of a single harmonic base signal, a logistic function \ref{eq:logistic} for decoherence and a gaussian noise distribution wiht an RMS of ten percent of the signal amplitude. 

\begin{equation}
	f_{logistic} = \frac{1}{1 + e^{-k(x-x_o)}}
	\label{eq:logistic}
\end{equation}

The first evaluation was on the convergence of the various methods. Figure \ref{fig:baseConv} shows the tune measurment for the various algorighms with increasing length of the sample window. It is clear to see the limitations of the FFT resolution from the availble signal. 

\begin{figure}
	\centering
	\includegraphics[width=0.5\linewidth]{placeholder.pdf}
	\caption{Convergence of tune measurments with signal window}
	\label{fig:baseConv}
\end{figure}

This can be improved by padding the signal with zeros. This serves to increase the resolution of the FFT and by extension the Jacobsen interpolation. We can verify that this improves the resolution of the, and logically has relatively negligible effects on the NAFF convergence.

\begin{figure}
	\centering
	\includegraphics[width=0.5\linewidth]{placeholder.pdf}
	\caption{Convergence of tune measuremnts with zero padding}
	\label{fig:padConv}
\end{figure}

As we can see in both cases, the NAFF convergence rate is superior even with the padding. There is an additional consideration, we can evaluate the accuracy of the measurments while adjusting the inital phase. There is a periodic variation of the measured tune for this change in phase that strongly depends on the length of the tune measuremnt window. Figure \ref{fig:baseRoll} shows the change in the tune measurments for varying the initial turn of the sample window. 

\begin{figure}
	\centering
	\includegraphics[width=0.5\linewidth]{placeholder.pdf}
	\caption{Variaiton of tune measurments with inital phase}
	\label{fig:baseRoll}
\end{figure}

The different tune measurement algorithms have different characteristic statistical uncertianties that scale with the availible sample window. Without interpolation, the uncertianty in the FFT lines goes as $\frac{1}{N}$ where N is sample points, in our case available turns.

The magnitude of this variation decreased for longer sample windows, but did not go below the statistical uncertianty of the methods. As a result the magnitude of this variation was added to the uncertianty of the tune measurements.


The same evaluations can be made on the real tbt BPM signals, but of course, the actual tune is unknown. In this case, the same analyses are considered to verify the same behaviours are preserved. Figure \ref{fig:measConvRoll} shows the same analysis for and example real kick response.

\begin{figure}
	\centering
	\includegraphics[width=0.5\linewidth]{placeholder.pdf}
	\caption{Tune measurment convergence and variation with phase for measured TBT data}
	\label{fig:measConvRoll}
\end{figure}


Multiple BPM signals are available and we would like to combine them, there were two approaches considered. The first is a bpm stacking approach, where a quasi-periodic signal with jumbled overlapped phases is constructed \cite{zisopolous}. This approach is unable to differentiate the sympathetic frequency components from large orthogonal kicks. 

The prefered approach to investigating multiple BPM response was to select the dominat components of a PCA decomposition for all combined BPM signals. Since we know the direction of the particular BPM buttons, we can differentiate horizontal and vertical signals, even if one kick dominated. NAFF could then be applied to the respective componenets to 

Another effect on tune measurements is resonant capture, where a small fraction of the beam becomes trapped in a resonant condition and continues to oscillate at the characteristic frequency of the resonance. As the rest of the beam quickly decoheres due to tune spread, theis dominates tune measurements with large sample windows. This can be seen in the spectrogram with short vertical decoherence with a long faint line. Resonant capture was additionally demonstrated on similar lines as seen in experiment in simulation with realistic bunches. Figure \ref{fig:simResCap} shows the tune space plot of the individually evaluated tunes of the macroparticles in a simulation. A low population component of the bunc has been trapped on a sextupole diffference line and coherently oscillates at this frequency while the rest of the bunch shows the typical strong amplitude dependent detuning. 

