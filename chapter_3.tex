\chapter{Experimental Analysis} \label{chap:analysis}


\section{Tune Measurements} \label{sec:tune}
Tune measurments provide long term coherent information on the dynamics and the strong dependence on amplitude is the chief experimental indicator of nonlinearities in the focusing. While decoherence limits the available turns for measurments, the variations in amplitude do not alter the fundamental frequency, which gives us a longer coherent signal. The tune was previously defined only for the linear system in the context of phase advance. The definition of the tune is more nebulous then for a strongly nonlinear system, as the motion is neccesarially anharmonic. For nonlinear dynamical systems, the poincare rotation number is a related quantity \cite{nagaitsevBetatron}, but difficult to relate to measurable quantities. For the experimental measurments, we interpret the dominant frequency of the transverse oscillation spectrum as the tune. 

Different tune measurment algorigthim were considered for the particular limitations of the available data. The simplest approach is simply taking the Fourier transform of turn by turn (TBT) BPM data and picking the peak frequency. As we have discrete sampling of the motion, the fast fourier transform (FFT) is the default approach. There are a number of methods to improve the resolution of the frequency measurements. In principle a windowing function can be applied to improve resolution of the peak frequency at the cost of supressing sidebands. In the case of the short coherent measurments, the reduction in available signal was more detrimental than the potential advantages. To bypass some of the limitations on the disctrete sample resolution of the FFT, Jacobsen interpolation \cite{jacobesnLocla} applies a quadratic interpolation to the FFT peak and its nearest neighbors for finer peak resolution. The implmentation used is based on that from the PyLHC \cite{cernomc} analysis library. The other approach considered is the Numerical Analysis of Fundamental Frequencies (NAFF) proposed by Lasker for the evaluation of long term stability of planetary systems \cite{laskar}. NAFF seeks to iteratively optimize the magnitude of a single fourier transform, or equation \ref{eq:naff}. In principle this is used to extract successive harmonics of the motion, but we are interested only in the dominant frequancy, so typically only one term is considered. The implementation of NAFF used in this analysis is PyNAFF \cite{zisopolusPyZ}. In addition to the offline methods, live tune measurments on circulating beam used a least squares approach, where an assumed functional form is fit to the tune, phase, coupling, and decoherence times.

\begin{equation}
	\Psi(T) = \frac{1}{2T}\int_{-T}^{T}f(t)e^{-i\omega t}dt
	\label{eq:naff}
\end{equation}

To evaluate the methods, a comparison was made with both a synthetic signal and the simulated bunch centroid signal from an Impact-X simulation. Figure \ref{fig:synTune} shows the evaluation signals, the amplitude does not impact the tune evaluation. The synthetic signal consisted of a convolution of a single harmonic base signal, a logistic function \ref{eq:logistic} for decoherence and a gaussian noise distribution wiht an RMS of ten percent of the signal amplitude. 

\begin{equation}
	f_{logistic} = \frac{1}{1 + e^{-k(x-x_o)}}
	\label{eq:logistic}
\end{equation}

\begin{figure}
	\centering
	\includegraphics[width=0.5\linewidth]{placeholder.pdf}
	\caption{Evaluation signals, fully synthetic and simulated}
	\label{fig:synTune}
\end{figure}

The first evaluation was on the convergence of the various methods. Figure \ref{fig:baseConv} shows the tune measurment for the various algorighms with increasing length of the sample window. It is clear to see the limitations of the FFT resolution from the availble signal. 

\begin{figure}
	\centering
	\includegraphics[width=0.5\linewidth]{placeholder.pdf}
	\caption{Convergence of tune measurments with signal window}
	\label{fig:baseConv}
\end{figure}

This can be improved by padding the signal with zeros. This serves to increase the resolution of the FFT and by extension the Jacobsen interpolation. We can verify that this improves the resolution of the, and logically has relatively negligible effects on the NAFF convergence.

\begin{figure}
	\centering
	\includegraphics[width=0.5\linewidth]{placeholder.pdf}
	\caption{Convergence of tune measuremnts with zero padding}
	\label{fig:padConv}
\end{figure}

As we can see in both cases, the NAFF convergence rate is superior even with the padding. There is an additional consideration, we can evaluate the accuracy of the measurments while adjusting the inital phase. There is a periodic variation of the measured tune for this change in phase that strongly depends on the length of the tune measuremnt window. Figure \ref{fig:baseRoll} shows the change in the tune measurments for varying the initial turn of the sample window. 

\begin{figure}
	\centering
	\includegraphics[width=0.5\linewidth]{placeholder.pdf}
	\caption{Variaiton of tune measurments with inital phase}
	\label{fig:baseRoll}
\end{figure}

The different tune measurement algorithms have different characteristic statistical uncertianties that scale with the availible sample window. Without interpolation, the uncertianty in the FFT lines goes as $\frac{1}{N}$ where N is sample points, in our case available turns.

The magnitude of this variation decreased for longer sample windows, but did not go below the statistical uncertianty of the methods. As a result the magnitude of this variation was added to the uncertianty of the tune measurements.


The same evaluations can be made on the real tbt BPM signals, but of course, the actual tune is unknown. In this case, the same analyses are considered to verify the same behaviours are preserved. Figure \ref{fig:measConvRoll} shows the same analysis for and example real kick response.

\begin{figure}
	\centering \includegraphics[width=0.5\linewidth]{placeholder.pdf}
	\caption{Tune measurment convergence and variation with phase for measured TBT data}
	\label{fig:measConvRoll}
\end{figure}


Multiple BPM signals are available and we would like to combine them, there were two approaches considered. The first is a BPM stacking approach, where a quasi-periodic signal with jumbled overlapped phases is constructed \cite{zisopolous}. This approach is unable to differentiate the sympathetic frequency components from large orthogonal kicks. 

The prefered approach to investigating multiple BPM response was to select the dominat components of a PCA decomposition for all combined BPM signals. Figure \ref{fig:pcaComps} shows the matrix resulting from the convolution of the BPM-specific terms in the PCA with the relative weights. Each column corresponds to a BPM, with the center line indicating the split between the horizontal and vertical BPM labels. 

\begin{figure}
	\centering
	\includegraphics[width=0.5\linewidth]{placeholder.pdf}
	\caption{PCA decomposition of TBT signals }
	\label{fig:pcaComps}
\end{figure}

The rows indicate different temporal signals. To provide some insight, in the case of simple harmonic motion we would expect four terms, two componetns for each phase in each plane. Each BPM signal is then composed of the relevant fraction of these phase components with scaling proporitonal to the beta function at that location. Figure \ref{fig:pcaTemps} shows the first four temporal components of the same decomposition. 

\begin{figure}
	\centering
	\includegraphics[width=0.5\linewidth]{placeholder.pdf}
	\caption{First four PCA temporal modes from decomposition above}
	\label{fig:pcaTemps}
\end{figure}

We can then discriminate on the direction of a tune component by looking at the relative magnitudes of the temporal components in the horizontal and vertical BPMs. A tune measurement technique, NAFF in this case, can then be applied to these individual components. This has the slight drawback of making phase measurments of individual BPMs more difficult, but this is not typically considered the the analysis anyways. 

Another effect on tune measurements is resonant capture, where a small fraction of the beam becomes trapped in a resonant condition and continues to oscillate at the characteristic frequency of the resonance. As the rest of the beam quickly decoheres due to tune spread, this dominates tune measurements with long sample windows. This can be seen in the spectrogram with short vertical decoherence with a long faint line. Figure \ref{fig:specResCap} shows such a spectogram and the associated kick profile. Here the vertical decoherence is very fast, as there is not much vertical kicked amplitude but large tune spread from the horizontal excitation. However, we see a faint line with long coherence right on the third order difference line $Q_x-2Q_y = 0$, which dominates the long window tune measurment.

\begin{figure}
	\centering
	\includegraphics[width=0.5\linewidth]{placeholder.pdf}
	\caption{Vertical Spectogram and Kick Profile}
	\label{fig:specResCap}
\end{figure}

Resonant capture was additionally demonstrated on similar lines as seen in experiment in simulation with realistic bunches. Figure \ref{fig:simResCap} shows the tune space plot of the individually evaluated tunes of the macroparticles in a simulation. A low population component of the bunch has been trapped on a sextupole diffference line and coherently oscillates at this frequency while the rest of the bunch shows the typical strong amplitude dependent detuning. 

\begin{figure}
	\centering
	\includegraphics[width=0.5\linewidth]{placeholder.pdf}
	\caption{Simulated resonant capture on same line}
	\label{fig:simResCap}
\end{figure}

The impact of resonant capture on tune measurments is that we prefer very short windows to capture the characteristic tune of the majority of the beam before it decoheres and becomes dominated by the resonant fraction.

\section{Nonlinear Insert Calibration} \label{sec:nioCal}
The nonlinear insert had to be calibrated and aligned to best match the nominal potential. This was done with beam based measurments. Recalling from section \ref{sec:dnMag} that the lowest order component of the nonliner insert is a quadrupole, and that the proper implementation of the DN NIO system requires consistent longitudinal scaling to properly match the potential, we treat the individual DN magnets as quadrupoles for small amplitudes. LOCO was applied to center the individual elements, the measured closed orbit offsets after centering \textit{assuming quadrupole terms} are presented in \ref{fig:dnOffset}. 

\begin{figure}
	\centering
	\includegraphics[width=0.5\linewidth]{placeholder.pdf}
	\caption{Closed orbit offests in DN magnet after manual alignment, assuming quadrupole terms}
	\label{fig:dnOffset}
\end{figure}


Each magnet was energized individually and a small amplitude kick was applied to the beam to measure the tune shift. This was done for multiple current setpoints to fit a tune shift vs current.
The individual scaling of the nonlinear t-parameter to current was calibrated for each magnet using the tune shift due to a quadrupole error in the lattice, (eq. \ref{eq:quadErrShift}) set equal to the quadrupole term of the DN potential (eq. \ref{eq:dnQuadTerm}), and solved for t.

\begin{equation}
    	\Delta Q_{x} = \pm \frac{1}{4\pi}\int{\beta_x(s) \frac{\Delta B_2}{B\rho} ds}
    	\label{eq:quadErrShift}
\end{equation}

\begin{equation}
	\Delta B_2 = \frac{-2B\rho \Delta t}{\beta^2(s)}
	\label{eq:dnQuadTerm}
\end{equation}

The resulting data gives a relative current scaling profile for the overall insert Fig. \ref{fig:dnIscaling}. This calibration assumes good accuracy of the beta functions in the nonlinear insert at the time of the calibration.

\begin{figure}
    \centering
    \includegraphics[width=0.5\linewidth]{placeholder.pdf}
    \caption{Relative DN insert current scaling profile compared to calculated ideal values}
    \label{fig:dnIscaling}
\end{figure}

This approach gives good relative scaling of the magnets, but we want to additionally evaluate the calibration of the insert as a whole. The tune shift assuming only the quadrupole terms of the DN element is given in Eq. (\ref{eq:dnDetune}) \cite{nagaitsevNonlinearOptics}.

\begin{equation}
	\begin{split}
	Q_{x} = Q_{o}\sqrt{1+2t} \\
        Q_{y} = Q_{o}\sqrt{1-2t}
	\end{split}
	\label{eq:dnDetune}
\end{equation}

The tune shift of the insert was measured by adjusting the t-parameter of the entire DN insert, according to the previously calibrated current ratios and kicking the beam to small amplitudes to measure the tune. The measured tunes are plotted with the quadrupolar detuning in Figure \ref{fig:dnDetuning}. The impact of the sextupoles used to compensate chromaticity can be seen inas the dominant effect on the tune near the third order resonances (red lines on tune diagram).

\begin{figure}
    \centering
    \includegraphics[width=0.5\linewidth]{placeholder.pdf}
    \caption{Measured nonlinear insert tune shift detuning}
    \label{fig:dnDetuning}
\end{figure}

The ratio of the tune shift between the planes is good indicator of proper implementation of the longitudinal scaling of the potenetial and beta function match in the insert. The absolute t-parameter scaling was calibrated by measuring tune shift vs nominal t-parameter, as presented in Figure \ref{fig:dnTuneVsT}. The gap in the data between t=0.2 t=0.3 was an unfortunate result of the BPM system freezing and failing to update TBT data, and was simply excluded.

\begin{figure}
    \centering
    \includegraphics[width=0.5\linewidth]{placeholder.pdf}
    \caption{Tune shfit vs t-parameter in both planes, before and after scaling}
    \label{fig:dnTuneVsT}
\end{figure}

The initial data indicated a discrepancy in the tune dependence of the t-parameter. The proportiona The ideal detuning expression was fit to the data with a t-scaling factor as the only free parameter Eq. (\ref{eq:fitTune}). 

\begin{equation}
    Q = Q_o \sqrt{1\pm 2at}
    \label{eq:fitTune}
\end{equation}

The fit resulted in a calibrated t-parameter scaling of $a \approx 0.935$. The source of this discrepancy is not clear, but may be a result of magnet crosstalk, as the relative calibration was measured with single elements.

\section{Momentum Reconstruction} \label{sec:momReconst}

\section{Kick Amplitude Calibration} \label{sec:kickAmpCal}

\section{Dynamic Aperture Evaluation} \label{sec:daEval}
The dynamic aperture (DA), or amplitude limit on stable motion, is a crucial metric of the real nonlinear system with full perturbations. The dynamic aperture is evaluated for simulated beams as the limit of asymptotically stable trajectories. For experimental measurements, the interpretation becomes more difficult. DA scans were performed by measuring the losses of kicked circulating beam. A consistent qualitative metric was applied, the loss limit was defined as a percent loss greater than 25\% or two subsequent kicks with percent losses greater than 10\%.

For a particular DA scan, a bunch was injected and scraped to a standard initial current in the BPM sensitivity range. The beam was iteratively kicked along fixed ratio "spokes" as current was logged. Once the current reached a minimum threshold, another bunch was injected and the scan of another spoke began.  Figure \ref{fig:spokes} illustrates the resulting geometry of the kick amplitudes.

\begin{figure}
	\centering
	\includegraphics[width=0.5\linewidth]{placeholder.pdf}
	\caption{Dynamic Aperture Scan configuration}
	\label{fig:spokes}
\end{figure}

Both the BPMs and DCCT were used to monitor beam current after a kick. The sum of the individual BPM button responses can be used as a fast indicator of losses. The BPM TBT output contains 20 turns of information before the kicker fires. For loss comparisons, the mean of the BPM signal for the first and last 20 turns was compared. The sample range of the BPMs was around 7000 turns which corresponds to a timescale of just under a millisecond. The DCCT signal was sampled before and after the kick on a timescale of a second. Loss timescales differed, some losses on kick were visible in the first few turns of the sum signal, and some losses were not visible in the sum signal but were visible in the DCCT. 

The noise in the DCCT signal was evaluated as the RMS noise with no circulating beam. The sum signal noise was evaluated as the RMS of the availible sample range.

To ensure that losses occurred in the time span between the DCCT samples, and were not dominated by the natural circulating beam lifetime, the difference between the second sample and the first sample of the next kick were calculated and verified to be at the level of noise in the DCCT signal. Figure \ref{fig:dcctKickLife}.


\begin{figure}
	\centering
	\includegraphics[width=0.5\linewidth]{placeholder.pdf}
	\caption{Change in DCCT current between DA scan kicks}
	\label{fig:dcctKickLife}
\end{figure}

Figure \ref{fig:daLimit} shows the DCCT and BPM sum signal for increasing kicks along a particular spoke. The red line indicates the determined DA limit for the particular spoke.

\begin{figure}
	\centering
	\includegraphics[width=0.5\linewidth]{placeholder.pdf}
	\caption{Percent Losses after individual kicks in a DA scan}
	\label{fig:daLimit}
\end{figure}

\section{Synchrotron Profiles} \label{sec:synchProfiles}
