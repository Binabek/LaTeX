\chapter{Integrable Optics Test Accelerator} \label{chap:iota}

\section{Facility Overview}

The Integrable Optics Test Accelerator (IOTA) was constructed at the Fermi National Accelerator Laboratory (Fermilab) for dedicated beam studies, particularly implementations of nonlinear integrable optics (NIO). IOTA is housed in the Fermilab Accelerator Science and Technology (FAST) facility and is not connected to the main sequence of proton accelerators at Fermilab. 

\section{IOTA Design}
IOTA shows significant design influence from the VEPP-2000 collider at the Budker Institute for Nuclear Physics \cite{vepp-2000}. IOTA is of a stretched octangonal layout with four thirty degree and four sixty degree primary dipoles forming the fundamental geometry. The dipole parameters are outlined in \ref{tab:dipole}.
