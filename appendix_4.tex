\chapter{Sextupole Errors} \label{apx:sextErr}

When optimizing the different sextupole configurations, it is important that the sextupoles are well aligned to minimize the impacts of the changing excitations to the linear lattice functions. This minimizes systematics in the evaluation due to chanages in relative beta function, path length, etc. For a static sextupole configuration, a round of LOCO can be run again and any remaining linear lattice impacts accounted for. Before running the optimization, each sextupole was excited to evaluate the closed orbit shift. A few sextupoles were found with outsized impacts on the closed orbit location. To increase the sensitivity of the relative measurements, a number of closed orbit bumps were constructed in the relevant sextupole locations. The relative closed orbit impacts vs excitation at the full range of these bumps was measured. Assuming local linearity of the fields, the effective dipole field was calculated for these bumps. By mapping these evaluated fields to the bump locations in the magnet, an approximate field mapping can be made of the element. 

Figre \ref{fig:unalignedSext3D} shows the evaluated ``normal" dipole field against the nominal orbit offset in an unaligned sextupole. Note the general saddle shape of the plot with positive quadradic dependence in $x$ and negative quadratic dependence in $y$. But, our magnetic center at the fixed point of the saddle is positioned well away from the center. While the closed orbit response to the excitations is noninar, the linear approximation of the fields would still preserve the same minimum of the quadratic dependence on bump positions. The center of quadratic fits of the fields vs the bump location indicates the offset to be mechanically shifted. Notably, a constant dipole term had to added to improve the goodness of fit to an acceptible level. This indicates a fundamental dipole error term in the relevant magnet.

\begin{figure}
	\centering
	\includegraphics[width=0.7\linewidth]{./appendix_figures/sd1lB13D.pdf}
	\caption{Measured field from closed orbit responses in an unaligned sextupole for different transverse closed orbit bumps. 2D projected points are the same measured fields with errors.}
	\label{fig:unalignedSext3D}
\end{figure}

The worst case alignment magnets were physically moved and the measurements repeated to evaluate the method. Figure \ref{fig:sextField3D} shows the same measumrents in an aligned sextupole. The saddle shape remains, but the fixed point is centered at the origin, demonstrating reasonable effectiveness of the approach. Only ``normal" dipole terms are presented as significant vertical orbit deviation was not an issue. This method produced reasonable results for the evaluated magnets, and clearly identified the elements with residual dipole terms, but was quite time consuming and sensitive to the calibration of the bumps in the magnets. 

\begin{figure}
	\centering
	\includegraphics[width=0.7\linewidth]{./appendix_figures/sd1lB13Dcentered.pdf}
	\caption{Measured field from closed orbit responses in an well aligned sextupole for different transverse closed orbit bumps. 2D projected points are the same measured fields with errors.}
	\label{fig:sextField3D}
\end{figure}

After the experiment, an example dipole which indicated residual dipole field was removed to be evaluated on the magnetic measurment stand described in \ref{apx:oct}. Excited measurments of the magnet indicated substantial dipole componentes after centering the multipole decomposition on the sextupole term, plotted in Figure \ref{fig:sextMult2A}.

\begin{figure}
	\centering
	\includegraphics[width=0.6\linewidth]{./appendix_figures/1AMultipoles.pdf}
	\caption{Multipole decomposition for sextupole with dipole error term.}
	\label{fig:sextMult2A}
\end{figure}

The resulting field map in Figure \ref{fig:sextMap} makes the residual dipole term clear as the field lines crossing the x-plane.

\begin{figure}
	\centering
	\includegraphics[width=0.6\linewidth]{./appendix_figures/1Amap.pdf}
	\caption{Measured field map for sextupole with dipole error. Lines indicate $B$ field direction, colors magnitude of field. Red is magnetic north pole, blue is south.}
	\label{fig:sextMap}
\end{figure}

To evaluate the impact of hysteresis, the magnet was deguassed by oscillating between positive and negative excitations with a steadily decreasing amplitude of excitation. After the degaussing procedure, the magnetic multipole composition was measured  without excitation and found to be on the order of the uncertianty of the hall probe. Exciting the magnet again reproduced the relevant dipole fields. This indicates a fundamental error in the magnet construction. Comparative simulations of the fields with deliberate errors were made in Mermaid (magnetic simulation software developed at Budker institute), consisting of single pole offests, magnet half misalignment. and missing turns. These simulations and mechanical measurements exhonerated simple mechanical misalignments as sufficient to cause the measured dipole term. The only remaining potential contributions are miswound coils or poles with significantly different relative permiability. The sextupoles in question were prototypes, and the neccesary vetting of steel permeability may not have been done. Regardless, the straightforward remedy for both errors are simply winding auxilary coils on the offending poles and powering them alongside to even the field.
