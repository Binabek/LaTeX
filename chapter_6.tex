\chapter{Summary and Conclusion} \label{chap:end}

The following is a brief summary of my major developments in experimental beam dynamics studies in IOTA and contributions to a deeper understanding of nonlinear integrable optics for particle accelerators. The experimental measurements in IOTA provided substantial new data on the NIO system, and motivated an extensive set of supporting simulations for further understanding of the dynamics.

Kicked beam measurements of the low emittance electron beam allowed turn-by-turn reconstruction of the dynamics in the accelerator. Fast decoherence due to strong amplitude dependent detuning from the NIO system complicated measurements of relevant quantities and required extensive verification of analysis algorithms. Supporting tracking simulations further verified that measurements of the centroid of the realistic IOTA beam closely matched the action of a single particle with identical amplitudes, even with strong nonlinearites. Virtual BPM reconstruction was implemented for analyzing the phase space and calibrating the amplitude of individual kicks. Initially, direct verification of conservation of the analytically predicted invariant quantities was attempted. The current IOTA configuration demonstrated clear variance in this quantity, especially with regards to the second invariant of the NIO system. Further investigating the regions of best conservation yielded results which indicated insufficient sensitivity to definitively demonstrate this conservation. Supporting tracking simulations with nonlineaities of the form expected in the bare IOTA lattice were similarly insensitive to this exact conservation evaluation consistent with the measured quantities. Amplitude dependent detuning measurements showed a significant tune footprint and impacts of perturbative nonlinearities in the form resonant capture of fractions of the bunch were well measured. Initial deviations from the predicted tune footprint were remedied with simulations including the expected residual nonlinearites.

The transverse aperture of the system was measured using kicked beam losses. The bare lattice aperture showed significant effects from the sextupole configuration in the lattice, and an effort to minimize this was undertaken. Evaluation of the aperture for different relative strengths of the integrable nonlinearites showed a slight reduction in aperture for a broad range of the strength, comparable to the aperture effects of the base sextupole configurations. For relative nonlinear strengths beyond two-thirds of the tune shift to the integer resonance, the aperture reduced significantly. Measurements of the aperture of the NIO system with various linear matching lattice perturbations demonstrated good robustness to a variety of knobs, indicating that the system is generally stable to typical lattice perturbations.

By increasing the relative strength of the integrable nonlinearities in the system, the expected shift in the working point to the vertical integer resonance was demonstrated. In this configuration, the unique nonlinear focusing dominates, and the circulating beam is not entirely lost. Nonlinear perturbations and imperfections in the implementation of the NIO potential means that the dynamic aperture shrinks substantially in this condition. However, the circulating beam within this aperture demonstrates measurable macroscopic lifetimes on the order of minutes. This corresponds to millions of turns and is indicative of asymptotic stability of particles within this aperture. This stability is only possible due to the nonlinear focusing from the NIO insert. Additionally, by further increasing the strength of the nonlinearites, the central fixed point on the closed orbit splits into two separate orbits which move vertically away from the original point. This was measured as two separate, stable ``beamlets" which moved apart with increasing nonlinear insert strength. This effect is a unique characteristic of the NIO system studied and is direct support of reasonable matching to the NIO condition.

Overall, strong evidence of the unique dynamics of the NIO system was observed, though detuning, the topology of the phase space, and unique stability and behaviour beyond the integer resonance. The system proved to be reasonably robust and stable in operation, even with significant linear and nonlinear perturbations. The strong detuning and reasonable aperture of the NIO system are promising steps towards implementation in an operational user-focused accelerator. Importantly, the dominant shortcomings of the current configuration in the form of strong sextupole perturbations were identified as a significant limiting factor on the NIO system, which will be leveraged in further experiments at IOTA.
