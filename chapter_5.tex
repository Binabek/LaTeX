\chapter{Simulation Studies} \label{chap:sims}

\section{Simulation Codes} \label{sec:simCodes}

There is a veritable zoo of commonly used codes for simulating the dynamics of charged particel accelerators, all with different focuses, operating assumptions, supported hardware, and dynamics models. A few different simulation softwares were used for supplementary simulation studies of IOTA and the NIO system. There were two main codes used, and three supplemental codes. The main lattice code for design and control of the linear optics was Six D Simulation (SixDSim), a java based application written and maintained by Alexsandr (Sasha) Romanov, the IOTA lattice expert. SixDSim performs lattice funciton calculations for periodic and channel mode accelerator systems in the full six dimensions, so first order dispersive and path lengthning effects are automatically considered. SixDsim provides general lattice construction and optimizing functionality with a graphical environment (a rarity in accelerator codes). In addition to offline lattice desing and manipulation, SixDsim has support for control system integration and was the main driving engine for the linear optimization of linear optics (LOCO) which defined the IOTA bare lattice for experiments. It also allowed for application of complicated custom knobs for fine tuning the lattice between time intensive LOCO iterations. As such, the SixDsim lattice model contained the best fit of the experimental IOTA lattice and was used for extraction of lattice funcitons used in calculations.

The other primary code used for IOTA simulations was Impact-X \cite{heubelNext}, a current generation particle tracking code with a particular focus on particle-in-cell space charge modelling and efficient operation on large scale computing environments. Impact-X was selected for two main reasons, the first is its implementation of the DN NIO lenses, and its advanced space charge models. The space charge simulation capibilites ended up being moot, as the proton program at IOTA was sufficeintly delayed to fall outside the scope of this thesis. The implementation of the DN lenses in Impact-X is based on the complex representation formulated in \cite{mitchellComplex}, which has some benifits in stability near the x-axis in configuration space. This is unlike other common tracking codes MAd-X and Xsuite which implement the original parameterization of the DN lens. Generally, Impact-X is a robust symplectic tracking code with modern Python scritping support and expanding physics cabibilities. The whole complement of tracking simulations presented were performed with Impact-X.

MAD-X is the closest the accelerator community has to a standard, and as such is the code in which the generic IOTA lattice is described. The cpymad wrapper for Mad-X was used as the benchmark for lattice functions in the tracking codes to evaluate proper lattice implementation, due to its straightforward implementation into python workflows. Xsuite is the new tracking code developed at CERN to replace and supplement the many legacy codes in use. For this work Xsuite was used for additional lattice benchmarking and as an educational tool. Finally, a self programmed python tracking code was used for initial evaluation of invariant conservation in various nonlinear insert configurations. This is a valuable educational experience, but not a reccomended long term approach. All of these simulation were re-performed with equivalent Impact-X lattices for better reproducibility and clarity.


