\chapter{Itroduction} 

\section{Background}
Charged particle accelerators have been productive experimental tools for fundamental physics experiments, from rutherfords use of natural acceleration of alpha particles for the discovery of the nucleus to the contemperary multinational physics collaborations centered at the Large Hadron Collider. "Maybe add other physics discoveries" In addition to their useful nature as tools, particle accelerators provide an interesting system to study dynamics in a controlled environment. These same dynamical studies then have direct impacts on the construction of new machines for fundamental studies. As the experiments drive increasing energy and power demands on the beams, careful control of losses becomes more important.

To tackle the challenge of controlling energetic subatomic particles we first have to choose our tools. Some of the earliest "accelerator" experiments relied on energetic decay products, leveraging the weak nuclear force. This is quickly limited by the characteristic energies of these decays. Gravity can be considered and easily dismissed as weak and effectively fixed on the earth's surface. Bulk material interactions can be used to affect the path of particles, but they cannot accelerate and tend to cause significant losses. The default choice has been electromagnetic fields, dating back to early cathode ray experiments. These fields are easy to produce and control and can both steer and provide energy to charged particles. Naturally this restricts our ability to work with neutral particle beams, but this can be typically overcome with a charged primary beam to produce a neutral secondary beam as is the case in succesful neutron spallation and neutrino beam facilities.

Scaling the electrostatic fields from our early cathode ray tubes are a straightforward starting point for accelerating particles. A large voltage applied across carefully shaped electrodes can simultaneously accelerate and focus a beam. This was the guiding principle for early Van De Graf machines like the Westinghouse atom smashe and the Cockroft-Walton style proton sources popularly used in the mid 20th century. This approach still finds use in pellotrons and "tandem" accelerators for low energy nuclear experimentation. However, the practical limits of breakdown gradients quickly limit the energy from such devices. Using an oscillating electric field is a practical approach to bypassing this limitation in two ways. First, the gradients of oscillaitng fields may be much higher than static fields. Second, by selectively timing the particles to be accelerated, the particle can gain energy from the gradient at the frequency of the oscillation. There are a few ways this is currently accomplished. The first is simply shielding the particle from a travelling electromagnetic wave, as is done in the Alverez or drift tube linear accelerator (linac). 

So far the problem of imparting energy to the particles has been given precendence, but naturally the particles must also be steered and focused for useful purposes. Here we can consider magnetic fields as well as electric. While not possible to be used for energy gain, magnetic fields are the preferred option for steering and control. While electrostatic fields are used in some low energy sections, the proportional scaling of the force with momentum from a magnetic field pay divedends as beam energy increases. 
