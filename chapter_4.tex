\chapter{Experimental Studies} \label{chap:expResults}

\section{NIO Studies Kicked Beam Collections}


\section{Working Point stability of Different NIO configurations} \label{sec:nioWorkPoint}
As discussed in section \ref{sec:nioConfig}, the NIO system requires chromatic compensation for stable operation. The base stability was investigated for different bare lattice configurations. Circulating beam was established and the nonlinear insert t-parameter was slowly ramped to establish stability. Figure \ref{fig:bareRamp} shows the ramp for the naieve bare lattice without sextupole compensation of any kind. We see complete losses near the horizontal third order resonance $3 Q_x = 1$. This may indicate residual sextupole components in the bare lattice. The source of these nonlinearities is not clear, but may stem from the dipole fringe fields, or the geometric nonlinearities from the tight bending radius. The source may also be destabilztion of the system due to adjustment of momentum.

\begin{figure}
	\centering
	\includegraphics[width=0.8\linewidth]{./chapter_4_figures/2023-10-15aVstSlowRamp.pdf}
	\caption{Circulating current in IOTA while ramping t-parameter for no sextupole compensation}
	\label{fig:bareRamp}
\end{figure}


Chromaticity was then compensated with a minimal set of two families of sextupoles. Figure \ref{fig:midRamp} shows a similar ramp after the compensation. The sextupole resonance is still present, but the stability is not strongly impacted and we can go all of the way to the integer resonance condition. We see a new loss location occur around $t=-0.46$, and some potential losses near $t=-0.3$. This motivates investigating the losses with a slowly varying t-parameter.

\begin{figure}
	\centering
	\includegraphics[width=0.8\linewidth]{./chapter_4_figures/2023-04-24aVstSlowRamp.pdf}
	\caption{Circulating current in IOTA while ramping t-parameter with chromatic compensation}
	\label{fig:midRamp}
\end{figure}


This slow ramp of the NIO insert is presented in Figure \ref{fig:slowRamp}. In this scan a second region of losses becomes more clear near the integer resonant condition. The nearest resonant line in tune space is $2Q_x + 2Q_y = 11$, in additon to a number of $6^{th}$ order lines. Whatever the cause of losses near this working point, it meant that this region had to be quicly stepped beyond when investigating dynamics at the integer resonance. Practically, the NIO insert was ramped to just above the limit of losses then "snapped" beyond this point. Notably there are no significat losses near $t=-0.30$.

\begin{figure}
	\centering
	\includegraphics[width=0.8\linewidth]{./chapter_4_figures/2023-05-02aVstSlowRamp.pdf}
	\caption{Circulating current in IOTA while slowly ramping t-parameter with chromatic compensation}
	\label{fig:slowRamp}
\end{figure}

Figure \ref{fig:fastRamp} shows quickly ramping does not incur significant losses. This is an improvement over the lattice tune in the previous runs. In the past, ad hoc sextupole knobs had to be ajdusted as the insert was ramped to stably operate at various tunes. However, this measurment was collected before the full DN calibration was finalized, so it does not quite arrive at the integer resonance equivalent setting.

\begin{figure}
	\centering
	\includegraphics[width=0.8\linewidth]{./chapter_4_figures/2023-04-24aVstMidRes.pdf}
	\caption{Circulating current in IOTA while quickly ramping t-parameter with chromatic compensation}
	\label{fig:fastRamp}
\end{figure}


\section{Conservation of Nonlinear Invariants} \label{sec:invConv}
A direct verification of proper implementation of the NIO system would be conservation of the analytically predicted invariants. The same approach was used in section \ref{sec:simInv} for evalution of insert configurations. To evaluate the invariant expressions, transverse momentum coordinates must be reconstructed as described in section \ref{sec:momReconst}, and the fitted coordinates needed to be normalized by the bare lattice Courant-Snyder functions at the virtual BPM location. Like the transfer matrices, these quantities were extracted from the design lattice fitted to by LOCO. To benchmark the method, the Courant-Snyder invariants were calculated for the same kicks. For reasonable comparison to the equivalent Courant-Snyder invariant, the first order effect on the lattice functions due to the NIO insert were simulated using quadrupoles in place of the full nonlinear elements. The calculations of the Courant-Snyder invariants used this normalization as opposed to the bare lattice functions. To avoid apparent invariant changes due to any linear coupling, the sum of the horizontal and vertical invariants (also the normalized Hamiltonian) was used for comparison \cite{leeAccelerator}.  Figure \ref{fig:invTbt} shows the resulting calculated invariant quantities TBT for two kicks. These parameters are normalized to the value at the first turn. Since the decoherence reduces the amplitude of the centroid signal, the invariant quantities also neccesarially reduce. The TBT BPM position uncertianty also impacts the invariant uncertianty. The left plot shows a good result for measured conservation, where both nonlinear predicted invariants show relatively small devation from their predicted values. The right plot shows a case where the second DN invariant is not conserved at all. We see large oscillations about a central value, this is a characteristic signature of an arbitrary value calculated from the dynamical variables. These responses were for the same lattice condition with different kick amplitues.

\begin{figure}
	\centering
	\includegraphics[width=1\linewidth]{./chapter_4_figures/goodBadInvConsTbt_t238.pdf}
	\caption{TBT calculated invariant quantities, left plot shows general good conservation of all invariant quantities, right plot shows poor conservation of DN second invariant}
	\label{fig:invTbt}
\end{figure}


An approach to evaluating the variance of the predicted invariants is to look at the frequency spectrum of the values. For a conserved quantity, we expect the TBT noise to dominate and the frequency spectrum to be flat. For a non-conserved quantity, we expect peaks in the frequency spectrum corresponding to the oscillation. Figure \ref{fig:invLogLog} shows the FFT spectra for the same kicks as in figure \ref{fig:invTbt}. As the decoherence generates a strong zero frequency term in the spectrum, a log-log scale is used to emphasize the higher frequency spectral peaks. We can see the clear peak in the spectrum for the poorly conserved second invariant. This approach is useful as a graphical method, but does not yield an easily interpritable reduced quantity for a given kick.

\begin{figure}
	\centering
	\includegraphics[width=1\linewidth]{./chapter_4_figures/fourierInvariantCons_t238.pdf}
	\caption{Turn based spectral composition of calculated invariant quantities for example sets}
	\label{fig:invLogLog}
\end{figure}

To compare different amplitudes and nonlinear insert configurations we calculate the standard deviation of these valuse over the first 28 turns. This was selected to maximize the number of points evaluated before the decoherence becomes prominant for the broadest range of sample kicks at the cost of significant uncertianty in the metric. The first 28 turns are highlighted in Figure \ref{fig:invTbt}. Figure \ref{fig:invt238} shows the variation of the Courant-Snyder Hamiltonian, DN Hamiltonian and DN second invariant over the first 28 turns versus the kicker setting. The lattice configuration is with the "lilac" sextupole configuration and a nominal $t=-0.238$. The coupled calibration factors are used, which results in a skewed grid from the inital cartesian input kicker settings.

\begin{figure}
	\centering
	\includegraphics[width=1\linewidth]{./chapter_4_figures/multitInvConsDeviation_t238.pdf}
	\caption{Color plot of standard deviation of analytically predicted invariants over first 28 turns, each plot has individual color scaling for better contrast but underlying metric is the same}
	\label{fig:invt238}
\end{figure}

The first DN invariant (the Hamiltonian) is similar in functional form to the overall Courant-Snyder invariant. As a result these two quantities display similar levels of conservation. A better metric is looking at the conservation of the second DN invariant. Based on this plotted invariant space we see reasonable conservation of both nonlinear invariants of motion for low vertical kicks and middling horizontal kicks. Based on this observation, a collection was taken for the same kicker settings repeatedly for the full beam lifetime. Once again, the sextupoles are in the "lilac" configuration and the nonliner $t=-0.238$. We can observe the nonlinear invariants versus current for this setting, in Figure \ref{fig:pointCurrent}. There is a slight reduction in overall conservation as current goes down, this is an expected effect of the decreasing signal to noise ratio we see in the BPM TBT signals. Additionally, the valuse are all within the statistical uncertianty for the range from about 0.33 to 0.5 mA.

\begin{figure}
	\centering
	\includegraphics[width=1\linewidth]{./chapter_4_figures/invConsVcurr_point_t238.pdf}
	\caption{Invariant conservation for repeated kicks with same nominal amplitude as current naturally decays}
	\label{fig:pointCurrent}
\end{figure}

This set corresponds to the best conservation of the nonlinear invariants via this method. We can compare to the predicted deviation between the conservation of the courant snyder invariant with adjusted lattice funcitons and the nonlinear invariants in simulation for the same initial amplitudes. Table \ref{tab:invConsv} shows the difference in the invariants in simulation and experiment. Unfortunately direct calculation of the analytical invariant quantities from fitted position data has insufficient resolution to demonstrate conservation of these invariants. We can still evaluate regions of better and worse conservation, but the direct quantitatve verificaiton is limited by the TBT noise in our BPM signals.

\begin{table}
    \centering
    \begin{tabular}{lcc}
        \toprule
        \textbf{Invariant} & \textbf{Simulation} & \textbf{Experimental}\\
        \midrule
        CS Hamiltonian & 1.9\% & 6.9\% \\
        DN Hamiltonian & 0.3\% & 5.6\% \\
        DN Invariant & 0.3\% & 6.9\% \\
        \bottomrule
    \end{tabular}
    \caption{Invariant conservation standard deviation for simulation and experiment with identical transverse initial conditions}
    \label{tab:invConsv}
\end{table}


\section{Amplitude Dependent Detuning} \label{sec:ampDetune}
A useful measurement of the nonlinearities of a system is to directly evaluate the change in tune with the amplitude. This is a figure of merit which impacts the potential effectiveness of the Landau damping of a system. The amplitude dependent detuning was measured for different t-parameter settings ranging from $t=-0.05$ to $t=-0.41$. The resonant capture effects had a strong impact on available detuning measurments beyond $t=-0.275$. Figure \ref{fig:t330Detuning} shows the detuning for a t-parameter setting of $t=-0.330$, note the clustering of tunes on the third order coupling line. The sextupole terms tend to drive this coupling and trap the beam on the resonance. This does not negatively impact stability, but renders us insensitive to the detuning effects from the NIO.

\begin{figure}
	\centering
	\includegraphics[width=1\linewidth]{./chapter_4_figures/zoomTuneSpreadPoint_t330.pdf}
	\caption{Amplitude dependent detuning for nominal $t=-0.330$}
	\label{fig:t330Detuning}
\end{figure}


Figure \ref{fig:t330phaseSpace} shows the resulting phase space for the beam on this coupling resonance. The contours are characteristic of the two to one ratio of the coupling tunes. While the amplitude dependent detuning from the nonlinear insert drives the beam centroid to the resonsnce, the resonance then dominates and we are insensitive to the NIO effects.

\begin{figure}
	\centering
	\includegraphics[width=0.8\linewidth]{./chapter_4_figures/coupledSextupolePhaseSpace.pdf}
	\caption{Reconstructed phase space for kicked beam captured on a second order coupling resonance}
	\label{fig:t330phaseSpace}
\end{figure}

The $Q_x = 1/3$ horizontal resonance line also impacted the availible aperture for tune measurments. Figure \ref{fig:t140phaseSpace} shows the reconstructed phase space of a kick near this line for a NIO setting at $t=-0.140$. The characteristic triangular phase space in the horizontal plane is visible.

\begin{figure}
	\centering
	\includegraphics[width=0.8\linewidth]{./chapter_4_figures/horizontalSextupolePhaseSpace.pdf}
	\caption{Reconstructed phase space for kicked beam with horizontal tune just above third order horizontal resonance}
	\label{fig:t140phaseSpace}
\end{figure}


Based on these constraints, to measure the maximum range of the detuning without resonant effects, a nominal working point of $t=-0.238$ was selected, which corresponds to $Q_x = 0.364, Q_y = 0.217$. Figure \ref{fig:t238detune45} shows the detuning for this point sampling 45 turns. This is quite fast and leads to some uncertianties in the tune measurments. Figure \ref{fig:t238detune45Error} shows a further zoomed version of the plot with errorbars indicating uncertianties of the tune measurments in each plane. The errors are vertically dominated as the decoherence in the vertical plane is much faster than in the horizontal plan. We still see resonant capture effects on fractions of the beam for this configuration. If we are dominated by TBT cntroid signal before full decoherence we are sensitive to the DN NIO detuning. However, if the tune length is sampled for a long span, resonant capture of a fraction of the bunch begins to dominate the tune measurments. Figure \ref{fig:t238detune190} shows the same tune space measurments for a TBT sample range of 190 turns. We see resonant capture near sextupole and octupole coupling lines.

\begin{figure}
	\centering
	\includegraphics[width=1\linewidth]{./chapter_4_figures/zoomTuneSpreadGrid_t238_45turn.pdf}
	\caption{Amplitude dependent detuning for nominal $t=-0.238$, sampled for 45 turns}
	\label{fig:t238detune45}
\end{figure}


\begin{figure}
	\centering
	\includegraphics[width=1\linewidth]{./chapter_4_figures/zoomTuneSpreadErrorbar_t238_45turn.pdf}
	\caption{Amplitude dependent detuning for same data as figure \ref{fig:t238detune45} with a tighter zoom to emphasize the main range of tunes with uncertianties. Vertical uncertianties dominate due to very short decoherence times}
	\label{fig:t238detune45Error}
\end{figure}

\begin{figure}
	\centering
	\includegraphics[width=1\linewidth]{./chapter_4_figures/zoomTuneSpreadPoint_t238_190turn.pdf}
	\caption{Amplitude dependent detuning for nominal $t=-0.238$, sampled for 190 turns. Resonant capture dominates tune measurments for a large fraction of tune space}
	\label{fig:t238detune190}
\end{figure}


We can also compare this detuning with our directly calibrated amplitude measurements. As the amplitude calibration depends on the linear lattice parameters, the direct amplitude dependent measurments were only valid for the "lilac" sextupole configuration. Figure \ref{fig:t238ampDetuneBase} shows the tune in both planes plotted agains the equivalent initial linear action in both planes. Each plot contains every kick so the perpendicular detuning and amplitude is visible as the low amplitude tune spread for a given combination. Simulated tunes for the ideal IOTA lattice are plotted alongside. The simulated tunes were calculated using initial positions set by the calculated positions from the measured tunes. The overall direction of the detuning and the ranges are comparable, but a number of features are lacking. The horizontal and vertical detuning versus the horizontal amplitude are suppressed, the slope of the detuning versus the veritcal kicks is different. We also see a slight mismatch in the origin of the detuning profiles, this indicates an imperfect bare lattice working point for the measument configuration. This makes direct comparison of detuning a little harder, but supports the general stability of the NIO insert.

\begin{figure}
	\centering
	\includegraphics[width=1\linewidth]{./chapter_4_figures/impactxRun4TuneVampComparison.pdf}
	\caption{Measured centroid tune versus fitted initial linear action with simulated tunes in idealized IOTA lattice for identical inital actions}
	\label{fig:t238ampDetuneBase}
\end{figure}


We know that the bare lattice is more complicated than the ideal situation. We can introduce the sextupole nonlinearities consistent with full compensation of the chromaticities, the exact dipole mappings, and the nonlinear quadrupole fringe field effects to the lattice (configuration S2 from section \ref{sec:iotaSim}). Figure \ref{fig:t238ampDetuneError} shows the same tune vs amplitude plots where the simulation contains these nonlinearities. The most immediately striking effect is the reduction in the range of the tune spread versus the horizontal amplitude, and the "folding" effect which more closely matches the tune ranges.

\begin{figure}
	\centering
	\includegraphics[width=1\linewidth]{./chapter_4_figures/impactxRun4TuneVampComparisonErrors.pdf}
	\caption{Measured centroid tune versus fitted initial linear action with simulated tunes in IOTA lattice with ad-hoc nonlinearities for identical inital actions}
	\label{fig:t238ampDetuneError}
\end{figure}

We make one more comparison, that of the detuning with the residual nonlinearities and only the quadrupole terms of the nonlinear insert to ensure that the dominant contribution to the detuning is the NIO insert. The quadrupole terms of the nonlinear insert must be simulated in order to match the relevant working point. In Figure \ref{fig:t238ampDetuneQuad} the same detuning vs amplitude plots show significant deviation favoring experimental measurment of the design NIO system. The detuning is much smaller and the horizontal detuning is anti-correllated and depends on the perpendicular amplitude compared th the DN detuning.

\begin{figure}
	\centering
	\includegraphics[width=1\linewidth]{./chapter_4_figures/impactxRun4TuneVampComparisonErrorsDNquad.pdf}
	\caption{Measured centroid tune versus fitted initial linear action with simulated tunes in IOTA lattice with ad-hoc nonlinearities and first order component of DN NIO insert for identical inital actions}
	\label{fig:t238ampDetuneQuad}
\end{figure}

The effect is more striking in tune space, Figure \ref{fig:t238detuneQuad} shows the simulated tune spread for configuration S3. Here we can clearly see the anti-correllated horizontal detuning across the opposite side of the DN tune shift.

\begin{figure}
	\centering
	\includegraphics[width=1\linewidth]{./chapter_4_figures/impactx_t238_tunes_chromSexts_exactBends_quadFringe_dnQuad.pdf}
	\caption{Amplitude dependent detuning for simulated IOTA lattice with ad-hoc nonlinearities and first order component of DN NIO insert}
	\label{fig:t238detuneQuad}
\end{figure}

We see significant impacts of the residual nonlinearities in the bare IOTA lattice. We can directly evaluate the amplitude dependent detuning for the bare "lilac" as well. Figure \ref{fig:t0detune} shows this detuning. We see coherent detuning along the linear coupling resonance.

\begin{figure}
	\centering
	\includegraphics[width=1\linewidth]{./chapter_4_figures/zoomTuneSpreadPoint_t000_190turn.pdf}
	\caption{Amplitude dependent detuning for bare IOTA lattice}
	\label{fig:t0detune}
\end{figure}

\section{Dynamic Aperture} \label{sec:DA}
To evaluate the different sextupole configurations, full aperture scans of the bare lattice were taken for different sextupole configurations after the optimization. \ref{fig:bareDA}

\begin{figure}
	\centering
	\includegraphics[width=0.6\linewidth]{./chapter_4_figures/sextupoleLossPhysical.pdf}
	\caption{Aperture limits of different sextupole configuratons for the IOTA bare lattice}
	\label{fig:bareDA}
\end{figure}

The dynamic aperture was measured for different lattice configurations and nonlinear insert settings. The losses along fixed amplitude ratio "spoke" kicks were taken as outlined in section \ref{sec:DAmethod}. To evaluate the loss amplitude we compare the initial amplitude to the effective linear action in the center of the nonlinear insert, as we expect this to be the minimizing aperture. In this approximaiton, the aperture comparison is seen in \ref{fig:iotaMinApDA}. We see reasonable aperture conservation in the vertical plane for the first few t-parameters, but the horizontal aperture shrinks to a plateau.

The measured losses with amplitude has a few caveats. An error in the admittance model of IOTA means we were not strictly limited by the small aperture in the center of the nonlinear insert.

Additionally, an effect related to the configuration space of the NIO system may shrink the admittance. For larger t-parameters, the configuration space begins to take on an "hourglass" shape, so for a given horizontal axis positon, parts of the beam off the median symmetry line may have larger horizontal offsets and be lost on the minimizing aperture.

\section{Nonlinear Stability at Integer Resonance} \label{sec:intCross}
The synchrotron camera images were used for evaluation of beam stability at the integer resonance condition. To make this observation, the t-parameter was incremented while logging the time dependence of the synchrotron radiation profiles. Based on the calibration of the t-strength from the linear working point in section \ref{sec:dnCal}, settings crossing the vertical integer resonance were selected. The linear system is fundamentally unstable at this tune, but the NIO system retains stability . Measurments supporting stability in these conditions are strong evidence for the NIO system. To evaluate stability, the lifetime of the beam was evaluated at each t-parameter setting. Beam current was low due to a restricted dynamic aperture and normal beam lifetime, so the DCCT measurments were insensitive and intensity was evaluated from the synchtrotron images. Section \ref{sec:synchLife} describes the methods to extract the lifetime from the synchtrotron images.

The topography of the DN system is useful here in tuning the measurments and determining the working point. At strengths of the nonlinear insert beyond the integer resonance, the origin of the system becomes an unstable fixed point. Two new stable fixed points are present above and below the origin. The result in the machine is the slow splitting of the beam into two stable ”beamlets” about these new fixed points. By evaluating when the beam becomse fully split, an upper limit on the crossing of the integer resonance location can be set. In addition to evaluating the integer resonance, the synchrotron profiles could be used for fine tuning the closed orbit throught the insert. Varying the closed orbit knobs to arrive at an equal distribution of beam in the top and bottom beamlets is a  clear indicator of proper alignment about the center of the insert. Figure \ref{fig:synchCenter} shows a comparison of the synchrotron profile of the beam before and after the centering.

The camera exposure had to be changed during the course of the measurements, as continuing beam loss reduced the signal. This means that only the relative intensity for a given exposure setting could be evaluated, and the absolute circulating current values were not availible. 

The lifetime measurements on the order of minutes indicate asymptotic stability of particles at the integer resonance where the nonlinear focusing terms dominate.
