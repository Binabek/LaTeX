\chapter{Experimental Studies} \label{chap:expResults}

\section{Working Point stability of Different NIO configurations} \label{sec:nioWorkPoint}

\section{Conservation of Nonlinear Invariants} \label{sec:invConv}
To evaluate the invariant expressions, the raw fitted coordinates needed to be normalized by the bare lattice Courant-Snyder functions at the virtual BPM location. Like the transfer matrices, these quantities were extracted from the design lattice. To benchmark the method, the Courant-Snyder invariants were extracted from the bare lattice.

For reasonable comparison to the equivalent Courant-Snyder invariant, the first order effect on the lattice functions of the DN element was extracted from the design lattice. This effect comes from the quadrupole term of the non-linear field. To avoid apparent invariant changes due to any linear coupling, the sum of the horizontal and vertical invariants (the Hamiltonian) was used for comparison \cite{leeAccelerator}. The first DN invariant, the Hamiltonian of the system, is very similar to the CS Hamiltonian. As a result these two quantities display similar levels of conservation. A better metric is looking at the conservation of the second DN invariant.

Direct calculation of the analytical invariant quantities from fitted position data has insufficient resolution to demonstrate conservation of these invariants.

\section{Amplitude Dependent Detuning} \label{sec:ampDetune}
A useful measurement of the nonlinearities of a system is to directly evaluate the change in tune with the amplitude. This is the direct figure of merit which impacts the potential strenght of the Landau damping of a system.

\section{Dynamic Aperture} \label{sec:DA}
The dynamic aperture was measured for a number of different lattice configurations and nonlinear insert settings.

\section{Nonlinear Stability at Integer Resonance} \label{sec:intCross}
The synchrotron camera images were used for evaluation of beam stability at the integer resonance condition. To make this observation, the t-parameter was incremented while logging the time dependence of the synchrotron radiation profiles. Based on the calibration of the t-strength from the linear working point in section \ref{sec:dnCal}, settings crossing the vertical integer resonance were selected. The linear system is fundamentally unstable at this location, but the NIO system retains stability for these settings, so measurments supporting stability in these conditions are fundamental evidence for the NIO system. To evaluate stability, the lifetime of the beam was evaluated at each t-parameter setting. Beam current was low due to a restricted dynamic aperture and normal beam lifetime, so the DCCT measurments were insensitive and intensity was evaluated from the synchtrotron images. Section \ref{sec:synchLife} describes the methods to extract the lifetime from the synchtrotron images.

The topography of the DN system is useful here in tuning the measurments and determining the working point. As the integer resonance is approached, the central fixed point becomes unstable and splits into two seperate fixed points.

Two runs were performed, one with the nominal lattice and one adjusted to best center the profile in the nonlinear potential.

The camera exposure had to be changed during the course of the measurements, as continuing beam loss reduced the signal.

Past the integer resonance condition the central stable fixed point becomes unstable, and two new stable fixed points move away from the origin with t-strength.

We can define an upper limit on the integer resonance crossing when these fixed points split and we see separate maxima in the beam.

Lifetime measurements indicate asymptotic stability of particles at the integer resonance where the nonlinear focusing terms dominate.
