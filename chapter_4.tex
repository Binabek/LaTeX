\chapter{Experimental Studies} \label{chap:expResults}

\section{Working Point stability of Different NIO configurations} \label{sec:nioWorkPoint}
As discussed in section \ref{sec:nioConfig}, the NIO system requires chromatic compensation for stable operation. The base stability was investigated for different bare lattice configurations. Circulating beam was established and the nonlinear insert t-parameter was slowly ramped to establish stability. Figure \ref{fig:bareRamp} shows the ramp for the naieve bare lattice without sextupole compensation of any kind. We see complete losses near the horizontal third order resonance $3 Q_x = 1$. This may indicate residual sextupole components in the bare lattice. The source of these nonlinearities is not clear, but may stem from the dipole fringe fields, or the geometric nonlinearities from the tight bending radius. The source may also be destabilztion of the system due to adjustment of momentum.

Chromaticity was then compensated with a minimal set of two families of sextupoles. Figure \ref{fig:midRamp} shows a similar ramp after the compensation. The sextupole resonance is still present, but the stability is not strongly impacted.

Figure \ref{fig:fastRamp} shows quickly ramping by these points does not incur significant losses. This is an improvement over the lattice tune in the previous runs. In the past, ad hoc sextupole knobs had to be ajdusted as the insert was ramped to stably operate at various tunes.

Another informative scan was done with a very slow ramp of the NIO insert in Figure \ref{fig:slowRamp}. In this scan a second region of losses becomes more clear near the integer resonant condition. The nearest resonant line in tune space is $Q_x + Q_y = 0$, but there are not elements expected to drive this term in the lattice. Whatever the cause of losses near this working point, it meant that this region had to be quicly stepped beyond when investigating dynamics at the integer resonance. Practically, the NIO insert was ramped to just above the limit of losses then "snapped" beyond this point.

\section{Conservation of Nonlinear Invariants} \label{sec:invConv}
To evaluate the invariant expressions, the raw fitted coordinates needed to be normalized by the bare lattice Courant-Snyder functions at the virtual BPM location. Like the transfer matrices, these quantities were extracted from the design lattice. To benchmark the method, the Courant-Snyder invariants were extracted from the bare lattice. 

For reasonable comparison to the equivalent Courant-Snyder invariant, the first order effect on the lattice functions of the DN element was extracted from the design lattice. The calculations of the Courant-Snyder invariants used this normalization as opposed to the bare lattice functions. To avoid apparent invariant changes due to any linear coupling, the sum of the horizontal and vertical invariants (also the normalized Hamiltonian) was used for comparison \cite{leeAccelerator}. The first DN invariant, the Hamiltonian of the system, is very similar to the CS Hamiltonian. As a result these two quantities display similar levels of conservation. A better metric is looking at the conservation of the second DN invariant.

Direct calculation of the analytical invariant quantities from fitted position data has insufficient resolution to demonstrate conservation of these invariants.

\section{Amplitude Dependent Detuning} \label{sec:ampDetune}
A useful measurement of the nonlinearities of a system is to directly evaluate the change in tune with the amplitude. This is the figure of merit which impacts the potential strenght of the Landau damping of a system. To measure the maximum range of the detuning without resonant effects, a nominal working point of $t=0.238$ was selected, which corresponds to $Q_x = , Q_y = $. As the amplitude calibration depends on the linear lattice parameters, the amplitude dependent measurments were only valid for the "lilac" sextupole configuration. 

\section{Dynamic Aperture} \label{sec:DA}
The dynamic aperture was measured for a number of different lattice configurations and nonlinear insert settings.

\section{Nonlinear Stability at Integer Resonance} \label{sec:intCross}
The synchrotron camera images were used for evaluation of beam stability at the integer resonance condition. To make this observation, the t-parameter was incremented while logging the time dependence of the synchrotron radiation profiles. Based on the calibration of the t-strength from the linear working point in section \ref{sec:dnCal}, settings crossing the vertical integer resonance were selected. The linear system is fundamentally unstable at this tune, but the NIO system retains stability . Measurments supporting stability in these conditions are strong evidence for the NIO system. To evaluate stability, the lifetime of the beam was evaluated at each t-parameter setting. Beam current was low due to a restricted dynamic aperture and normal beam lifetime, so the DCCT measurments were insensitive and intensity was evaluated from the synchtrotron images. Section \ref{sec:synchLife} describes the methods to extract the lifetime from the synchtrotron images.

The topography of the DN system is useful here in tuning the measurments and determining the working point. At strengths of the nonlinear insert beyond the integer resonance, the origin of the system becomes an unstable fixed point. Two new stable fixed points are present above and below the origin. The result in the machine is the slow splitting of the beam into two stable ”beamlets” about these new fixed points. By evaluating when the beam becomse fully split, an upper limit on the crossing of the integer resonance location can be set. In addition to evaluating the integer resonance, the synchrotron profiles could be used for fine tuning the closed orbit throught the insert. Varying the closed orbit knobs to arrive at an equal distribution of beam in the top and bottom beamlets is a  clear indicator of proper alignment about the center of the insert. Figure \ref{} shows a comparison of the synchrotron profile of the beam before and after the centering.

The camera exposure had to be changed during the course of the measurements, as continuing beam loss reduced the signal.

Lifetime measurements indicate asymptotic stability of particles at the integer resonance where the nonlinear focusing terms dominate.
